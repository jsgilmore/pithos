\documentclass[10pt,a4paper,conference]{IEEEtran}

% Some very useful LaTeX packages include:
% (uncomment the ones you want to load)

% *** CITATION PACKAGES ***
%
\ifCLASSOPTIONcompsoc
  % IEEE Computer Society needs nocompress option
  % requires cite.sty v4.0 or later (November 2003)
  \usepackage[nocompress]{cite}
\else
  % normal IEEE
  \usepackage{cite}
\fi
% cite.sty was written by Donald Arseneau
% V1.6 and later of IEEEtran pre-defines the format of the cite.sty package
% \cite{} output to follow that of IEEE. Loading the cite package will
% result in citation numbers being automatically sorted and properly
% "compressed/ranged". e.g., [1], [9], [2], [7], [5], [6] without using
% cite.sty will become [1], [2], [5]--[7], [9] using cite.sty. cite.sty's
% \cite will automatically add leading space, if needed. Use cite.sty's
% noadjust option (cite.sty V3.8 and later) if you want to turn this off.
% cite.sty is already installed on most LaTeX systems. Be sure and use
% version 4.0 (2003-05-27) and later if using hyperref.sty. cite.sty does
% not currently provide for hyperlinked citations.
% The latest version can be obtained at:
% http://www.ctan.org/tex-archive/macros/latex/contrib/cite/
% The documentation is contained in the cite.sty file itself.
%
% Note that some packages require special options to format as the Computer
% Society requires. In particular, Computer Society  papers do not use
% compressed citation ranges as is done in typical IEEE papers
% (e.g., [1]-[4]). Instead, they list every citation separately in order
% (e.g., [1], [2], [3], [4]). To get the latter we need to load the cite
% package with the nocompress option which is supported by cite.sty v4.0
% and later. Note also the use of a CLASSOPTION conditional provided by
% IEEEtran.cls V1.7 and later.


% *** GRAPHICS RELATED PACKAGES ***
%
  \usepackage[pdftex]{graphicx}
  \graphicspath{{../Figures/}}
  \DeclareGraphicsExtensions{.pdf,.png}
  \usepackage{color}

% *** MATH PACKAGES ***
%
\usepackage[cmex10]{amsmath}
% A popular package from the American Mathematical Society that provides
% many useful and powerful commands for dealing with mathematics. If using
% it, be sure to load this package with the cmex10 option to ensure that
% only type 1 fonts will utilized at all point sizes. Without this option,
% it is possible that some math symbols, particularly those within
% footnotes, will be rendered in bitmap form which will result in a
% document that can not be IEEE Xplore compliant!
%
% Also, note that the amsmath package sets \interdisplaylinepenalty to 10000
% thus preventing page breaks from occurring within multiline equations. Use:
%\interdisplaylinepenalty=2500
% after loading amsmath to restore such page breaks as IEEEtran.cls normally
% does. amsmath.sty is already installed on most LaTeX systems. The latest
% version and documentation can be obtained at:
% http://www.ctan.org/tex-archive/macros/latex/required/amslatex/math/

%\usepackage{amssymb}%............................ AMS Symbol fonts



% *** SPECIALIZED LIST PACKAGES ***
%
%\usepackage{algorithmic}
% algorithmic.sty was written by Peter Williams and Rogerio Brito.
% This package provides an algorithmic environment for describing algorithms.
% You can use the algorithmic environment in-text or within a figure
% environment to provide for a floating algorithm. Do NOT use the algorithm
% floating environment provided by algorithm.sty (by the same authors) or
% algorithm2e.sty (by Christophe Fiorio) as IEEE does not use dedicated
% algorithm float types and packages that provide these will not provide
% correct IEEE style captions. The latest version and documentation of
% algorithmic.sty can be obtained at:
% http://www.ctan.org/tex-archive/macros/latex/contrib/algorithms/
% There is also a support site at:
% http://algorithms.berlios.de/index.html
% Also of interest may be the (relatively newer and more customizable)
% algorithmicx.sty package by Szasz Janos:
% http://www.ctan.org/tex-archive/macros/latex/contrib/algorithmicx/

% *** ALIGNMENT PACKAGES ***
%
\usepackage{array}
% Frank Mittelbach's and David Carlisle's array.sty patches and improves
% the standard LaTeX2e array and tabular environments to provide better
% appearance and additional user controls. As the default LaTeX2e table
% generation code is lacking to the point of almost being broken with
% respect to the quality of the end results, all users are strongly
% advised to use an enhanced (at the very least that provided by array.sty)
% set of table tools. array.sty is already installed on most systems. The
% latest version and documentation can be obtained at:
% http://www.ctan.org/tex-archive/macros/latex/required/tools/


\usepackage{mdwmath}
\usepackage{mdwtab}
% Also highly recommended is Mark Wooding's extremely powerful MDW tools,
% especially mdwmath.sty and mdwtab.sty which are used to format equations
% and tables, respectively. The MDWtools set is already installed on most
% LaTeX systems. The lastest version and documentation is available at:
% http://www.ctan.org/tex-archive/macros/latex/contrib/mdwtools/

% IEEEtran contains the IEEEeqnarray family of commands that can be used to
% generate multiline equations as well as matrices, tables, etc., of high
% quality.

% *** SUBFIGURE PACKAGES ***
\ifCLASSOPTIONcompsoc
  \usepackage[caption=false,font=normalsize,labelfont=sf,textfont=sf]{subfig}
\else
  \usepackage[caption=false,font=footnotesize]{subfig}
\fi

%Setting captions to centered (Not IEEE journal standard)
%\makeatletter
%\long\def\@makecaption#1#2{\ifx\@captype\@IEEEtablestring%
%\footnotesize\begin{center}{\normalfont\footnotesize #1}\\
%{\normalfont\footnotesize\scshape #2}\end{center}%
%\@IEEEtablecaptionsepspace
%\else
%\@IEEEfigurecaptionsepspace
%\setbox\@tempboxa\hbox{\normalfont\footnotesize {#1.}~~ #2}%
%\ifdim \wd\@tempboxa >\hsize%
%\setbox\@tempboxa\hbox{\normalfont\footnotesize {#1.}~~ }%
%\parbox[t]{\hsize}{\normalfont\footnotesize \noindent\unhbox\@tempboxa#2}%
%\else
%\hbox to\hsize{\normalfont\footnotesize\hfil\box\@tempboxa\hfil}\fi\fi}
%\makeatother


% *** FLOAT PACKAGES ***
%
\usepackage{fixltx2e}
% fixltx2e, the successor to the earlier fix2col.sty, was written by
% Frank Mittelbach and David Carlisle. This package corrects a few problems
% in the LaTeX2e kernel, the most notable of which is that in current
% LaTeX2e releases, the ordering of single and double column floats is not
% guaranteed to be preserved. Thus, an unpatched LaTeX2e can allow a
% single column figure to be placed prior to an earlier double column
% figure. The latest version and documentation can be found at:
% http://www.ctan.org/tex-archive/macros/latex/base/

% *** PDF, URL AND HYPERLINK PACKAGES ***
%
\usepackage{url}

\usepackage{sistyle}
    \SIstyle{S-Africa}
    \SIunitspace{{\cdot}}
    \SIunitdot{{\cdot}}

% generate nice bookmarks and hyperrefs when exporting to pdf and dvi (screen version):
%\usepackage[a4paper,plainpages=false,colorlinks,linktocpage,bookmarks=true,bookmarksopen=false]{hyperref}
% use this for printing only (no color, print version):
%\usepackage[a4paper,plainpages=false,colorlinks=false,linktocpage,bookmarks=true,bookmarksopen=false]{hyperref}
% use this for conference papers where boxes will not look nice. (all colors=black, print version):
\usepackage[a4paper,plainpages=false,colorlinks=true, citecolor=black, filecolor=black, linkcolor=black, pdfhighlight=/O, urlcolor=black, linktocpage,bookmarks=true,bookmarksopen=false]{hyperref}

% correct bad hyphenation here
\hyphenation{op-tical net-works semi-conduc-tor}

%Add elegant support for Big-O notation
\providecommand{\OO}[1]{\operatorname{O}\left(#1\right)}

\begin{document}

%
% paper title
\title{Pithos: A State Persistency Architecture for Peer-to-Peer Massively Multiuser Virtual Environments}

\author{\IEEEauthorblockN{John S. Gilmore and Herman A. Engelbrecht\\
\IEEEauthorblockA{MIH Media Lab, Electrical and Electronic Engineering Department\\
University of Stellenbosch, Stellenbosch, South Africa\\
mail: jgilmore@ml.sun.ac.za and hebrecht@sun.ac.za}}}

\maketitle

\begin{abstract}
%\boldmath
An aspect of peer-to-peer (P2P) massively multiuser virtual environments (MMVEs) that has not received significant attention thus far, is how game
objects can be stored in a distributed fashion, taking into account the unique aspects of MMVEs. The field is termed P2P MMVE state persistency. This
paper explores the different models of state persistency for P2P MMVEs, namely super peer storage, overlay storage, hybrid storage and distance-based
storage. All storage types are found to be lacking in some respects and therefore a novel storage system called Pithos is introduced. Pithos provides
low latency storage for peers within the same group, while enabling higher latency, but reliable backup to a storage overlay. This is achieved by a
two-tiered hybrid architecture making use of player grouping.
\end{abstract}


\section{Introduction}
\label{introduction}

%P2P MMVE Background
Peer-to-Peer (P2P) Massively Multiuser Virtual Environments (MMVEs) have received significant attention from the research community, since the first
publication on the subject by Knutssonn et al. in 2004 \cite{knutsson_p2p_first}. P2P MMVEs promise to solve many issues prevalent in today's
Client/Server (C/S) based MMVEs.

%Key challenges and focus
Recently, six key challenges of P2P systems have been identified: \emph{interest management}, \emph{game event dissemination}, \emph{non-player
character (NPC) host allocation}, \emph{game state persistency}, \emph{cheating mitigation} and \emph{incentive mechanisms}
\cite{Fan_deisgn_issues_p2p}. Most of the challenges mentioned have received significant attention from the research community, with the exception of
state persistency.

State persistency defines how object states should be stored. Objects states can be anything from a user's position to the state of the virtual
market in an MMVE. For a P2P MMVE, game data must be distributed amongst various peers in the network. This creates challenges not usually present in
classic C/S MMVEs. We've identified the main requirements of P2P MMVE state persistency in \cite{gilmore_p2p_mmog_state_persistency} as scalability,
reliability, fairness, responsiveness and security.

In \cite{gilmore_p2p_mmog_state_persistency} we argue that none of the current approaches to state persistency satisfy all identified requirements.
The focus of this paper is, therefore, exclusively on state persistency in P2P MMVEs. It presents a design that satisfies all the identified
requirements, along with some implementation details and preliminary results. Pithos, a novel hybrid multi-tiered state persistency architecture is
proposed. The novelty of Pithos lies in its support for both a responsive and a fair storage system, while also taking into account security aspects
of distributed storage. To the best of our knowledge, there are some storage systems that provide responsive or fair storage, but none that provide
both. No storage system, designed specifically for P2P MMVEs, have taken security into account.

If Pithos is incorporated into an existing P2P MMVE network architecture, it will add the ability to reliability share the storage load of long and
short term game state. The addition of a robust state persistency mechanism, specifically designed for P2P MMVEs, will bring us one step closer to
the creation of a complete P2P MMVE architecture.

%TODO: This should be reviewed
%Summary
Section \ref{current_models} describes the different storage models found in the literature and describes the advantages and disadvantages of each
group.
%
Section \ref{design} presents the key design characteristics the define Pithos.
%
Section \ref{evaluation} describes the implementation of the simulated Pithos and continues to evaluate the responsiveness and fairness of the
simulation.
%
Section \ref{conclusion} concludes with a summary of the paper and a discussion on future work.

\section{Distributed state persistency models}
\label{current_models}

%Overview of four approaches
After reviewing various storage architectures, the reviewed architectures were categorised into four broad types by which state persistency is
currently achieved in P2P MMVEs \cite{gilmore_p2p_mmog_state_persistency}. The four identified storage types are: \emph{super peer storage},
\emph{overlay storage}, \emph{distance-based storage} and \emph{hybrid storage}. In order to evaluate the different storage types, the following
requirements were identified:

\emph{Scalability}: For an MMVE state persistency architecture to be scalable, it should be able to support thousands of users. In the paper,
scalability it not handled as some separate entity, but rather all other requirements are evaluated in terms of a system of thousands of users.

%Reliability
\emph{Reliability}: Reliability is defined to mean that a file in the storage system may neither be lost, nor be unavailable when a user requests it.

%Fairness
\emph{Fairness}: For a system to be fair, the responsibility of storage should be equally shared amongst all users according to their available
resources. This ensures that costs due to bandwidth and storage are shared amongst all.

%Responsiveness
\emph{Responsiveness}: Responsiveness is a requirement that has not really been part of file storage systems in the past. For MMVEs, it is believed
that responsive object storage is a key requirement to promote responsive game play and robust recovery mechanisms.

%Security
\emph{Security} Data security is a major issue for distributed storage, because data are stored on users' machines who should not necessarily be able
to access and alter the data stored.

The following sub-sections briefly describe the different storage types and lists the advantages and disadvantages of each. For a thorough evaluation
of the different storage types, the reader is referred to \cite{gilmore_p2p_mmog_state_persistency}.

\subsection{Super peer storage}

Super peer storage uses super peers to store the game state \cite{knutsson_p2p_first}. The game world is usually divided into multiple regions, with
the root objects of each region stored on different super peers. Each super peer stores the authoritive objects for its region. All peers in a region
access their regional super peer to store and retrieve regional object states. In this way, super peer storage can be seen as a C/S type of storage,
with super peers acting as regional servers. To improve reliability, multiple super peers per region are sometimes used, where each super peer in the
same region stores the same objects as all other super peers in that region \cite{varvello_p2p_second_life}.

The advantage of super peer storage is responsiveness. The disadvantages of super peer storage stem from the per-region centralised approach, where
super peers have access to all region data and they are the only entities in the network that store data. This leads to the main issues with super
peer storage being lack of fairness and security. It is also difficult to achieve high levels of reliability.

\subsection{Overlay storage}

%Description
Overlay storage distributes the game state across a P2P overlay \cite{Douglas05enablingmassively}, \cite{using_freenet_storage}, \cite{Fan_phd},
\cite{past_storage_focus}. Any available P2P overlay can be used, but a structured overlay is preferred because all objects are always accessible,
which is not always the case in a unstructured overlay. Various distributed storage techniques are compared in
\cite{Hasan_distributed_storage_survey}.

%Advantages/Disadvantages
Overlay storage is characterised by all nodes sharing all objects as well as multiple replicas of those objects. The storage time is $O(\log(N))$,
where $N$ is the total number of nodes in the network. Therefore, the main advantages of overlay storage are reliability, fairness and reasonable
security, with the main disadvantage being lack of responsiveness.

\subsection{Super peer-overlay hybrid storage}

%Description
Hybrid storage describes any combination of the other three storage types. Only one type of hybrid storage is, however, currently found in the
literature \cite{zoned_federation}, \cite{hybrid_storage1}, which is the combination of super peer and overlay storage.

Super peer-overlay hybrid storage divides the world into regions, as with super peer storage. Super peers are assigned region states to store, as
with super peer storage, but the complete region state is also stored using a form of overlay storage.

%Advantages/Disadvantages
Super peer-overlay hybrid storage combines some of the features present in overlay and super peer storage to have the advantages of high reliability
from overlay storage and high responsiveness from super peer storage. Hybrid storage, does however, still suffer from the same security issues and
lack of fairness present in super peer storage.

\subsection{Distance-based storage}
\label{classic_distance_based}

Distance-based approaches distribute objects to the closest peers in the virtual world \cite{colyseus_distance_based}, \cite{solipsis}. The
hypothesis is that peers are interested in objects that are close to them, so the object access latency can be reduced by storing objects on nodes
that are interested in them. Some storage approaches only consider player characters (PCs) for storage and do not consider non-player characters
(NPCs) \cite{individual_storage1}, \cite{cheat_proof_playout}. These storage types are considered sub-sets of distance-based storage. Some
specialised distance-based approaches exist that partition the virtual world into a Voronoi diagram to determine which objects are closest to which
peers \cite{Buyukkaya_voronoi_state_management}, \cite{Hu_voronoi_IM}.

%Advantages/Disadvantages
Distance-based storage exploits the assumptions that objects close to each other in the virtual world are more likely to interact with one another
and that the number of objects stored per nodes depends on the distribution of nodes in the virtual world. This leads to responsiveness being the
main advantage of distance-based storage. The disadvantages are a lack of reliability and security, and that it can suffer from some fairness issues.

\subsection{Summary and comparison}

From the above discussion, it is evident that none of the mentioned storage types in their current forms are appropriate for data storage in MMVEs.
Super peer storage is not fair nor secure, overlay storage is not responsive, hybrid overlay/super peer storage is not fair and distance-based
storage is not secure and not yet reliable.

\section{Architecture characteristics}
\label{design}

In this section, ``Pithos'', the proposed P2P MMVE state persistency architecture design is described. The inspiration for this architecture come
from two observations:
%
\begin{enumerate}
  \item One can combine multiple storage models and arrive at a model which possesses fewer disadvantages than any of the models used.
  \item Responsiveness is greatly increased in a fully distributed model, where there is no intermediate server that relays all information.
      However, fully distributed architectures are not scalable because the number of messages scaling by $O(N^2)$, where $N$ is the number of
      nodes in the network.
\end{enumerate}

\begin{figure}[htbp]
 \centering
 \includegraphics[clip=true, viewport=7.5cm 2.5cm 26cm 20cm, width=0.7\columnwidth]{CDHT_layout}
 \caption{Layout of the Pithos storage architecture}
 \label{fig_pithos}
\end{figure}
%
Figure \ref{fig_pithos} shows the Pithos architecture. The figure shows groups of fully connected peers (light blue and dark red), where all groups
are connected to each other in an P2P overlay through super peers (red).

Pithos groups peers to form a two tiered storage model. The first tier is a storage model at group level and the second is a storage model over all
groups. On the first tier, which is the intra-group level, a fully distributed storage system is used to allow for highly responsive read and write
operations within the group. On the second tier, which is the inter-group level, a P2P overlay is used to store data between groups.

According to categorisation of \ref{current_models}, Pithos is a type of hybrid storage, that incorporates overlay storage and distance-based
storage. Responsiveness is achieved by constructing fully connected networks amongst groups of players and then storing objects that are mostly used
by the group within the group, as described in Sections \ref{grouping}, \ref{store_retrieve} and \ref{distance_based}. Reliability is achieved by
making use of replication and migration mechanisms as described in Section \ref{store_retrieve}. Security is achieved by using a certification
authority to assign node IDs and signing any storage and retrieve request with the requesting node's certificate, as described in Section
\ref{secure_ids}. Fairness is achieved by having all nodes store objects, as described in Sections \ref{store_retrieve} and \ref{distance_based}.

\subsection{Grouping}
\label{grouping}

%Speak more concretely of grouping algorithms
At the core of the architecture is the peer clustering mechanism. Two approaches are being evaluated: distributed clustering techniques (for example
affinity propagation \cite{affinity_propagation}) and dynamic regioning techniques (for example self-organising spatial publish subscribe (SOSPS)
\cite{self_organising_sps_post}).

\emph{Distributed peer clustering techniques}: make use of the flocking behaviour of players to dynamically group players into flocks or clusters
\cite{flocking}. The main idea of flocking is that players move around in groups, rather than randomly on their own. It is desirable that user
density within groups should remain constant, because a fully distributed architecture is not scalable. This means that groups should merge or split
as the user density within them change.

Affinity propagation clusters nodes using a similarity matrix to find similar nodes. The similarity matrix may contain user positions. In this case,
affinity propagation will group nodes depending on their location in a virtual world. This algorithm is ideally suited to P2P applications, since it
is a distributed clustering algorithm based on message passing.

\emph{Dynamic regioning}: divides the virtual world into regions that can be resized or further divided to maintain constant player densities across
regions. SOSPS creates dynamic regions based on a Voronoi overlay network \cite{voronoi_diagrams_survey}. Near constant user density is achieved by
increasing and decreasing the area sizes. This system is based on VON, a distributed Voronoi overlay network designed for MMVEs \cite{VON_VAST}.

\subsection{Replication}
\label{store_retrieve}

When storing objects in Pithos, replication is used to increase object availability under network churn and for security in the presence of malicious
nodes \cite{storage_and_chaching_PAST}. For every object that is stored in Pithos, $k$ object replicas are also stored. The number of replicas ($k$)
depends on the degree of network churn as well as the number of expected malicious users in the network. If the network churn is high, more replicas
are required to avoid the situation where all $k$ peers hosting an object leaves the network before any object migration can be done.

If a node leaves the network and stops to transmit ``keep alive'' messages, the migration mechanism will detect this and replicate the file on
another node. Replication exists intra- as well as inter-group and is useful in ensuring that if a nodes leaves the network, the data are not lost.
All object requests are routed to the peer with the next closest ID if the root peer leaves, because of how overly routing functions. The new
destination peers will possess the stored files, since Pithos stores overlay replicas at overlay neighbours.

Another reason to replicate game objects is to make the system more secure. If it is known that a certain percentage of users are malicious, it is
advantages to have more replicas than malicious users. This will allow for a secure system where object hashes can be compared to determine which
nodes are malicious and what version of an object is accurate.

\subsection{Distance-based storage}
\label{distance_based}

For Pithos to succeed as an MMVE storage architecture, intra-group data requests should be preferred to inter-group data requests. This requirement,
combined with the fact that the grouping algorithm geographically groups players in the virtual world, lends Pithos to a storage system based on
distance-based storage. Similar to interest management, the assumption is that players have a limited area of interest and require interaction with a
limited number of objects within range.

Therefore, distance-based storage is implemented on a group level rather than an individual level. This means that objects are stored on the nearest
group of players, rather than the nearest user. It is assumed that such an approach will alleviate the security and reliability challenges present in
distance-based storage \cite{gilmore_p2p_mmog_state_persistency}.

With group-based distance-based storage, it is assumed that because peers now store objects closest to the group, the objects that they are
interested in will most likely be stored within their own group. Therefore, most data requests should be intra-group requests. The overlay storage
component ensures that nodes that require data, which are not stored within their group, are still able to access requested data.

\subsection{Secure storage and node ID assignments}
\label{secure_ids}

In order to design a secure distributed storage system, one requirement for the P2P overlay is that nodes should not be able to select their own IDs
or it will not be possible to secure the system against attack. Node IDs should rather be assigned securely by some certification authority
\cite{secure_overlay_routing}.

To meet this requirement, Pithos implements its own certification authority to assign node IDs securely and promote security in the P2P overlay. A
certification server exists that handle ID requests from nodes. The server assigns IDs to nodes and provides the node with a signed certificate that
it may use to store data.

Whenever an object is stored or updated in the storage network, nodes have to sign the object to enable the tracking of object changes throughout the
life of the object. This system is very different from classic distributed file storage designs that advocate anonymity in storage. The fact that all
changes can be tracked to a specific node will simplify the task of eliminating user cheating.

\section{Architecture evaluation}
\label{evaluation}

Pithos is currently still a work in progress, and although the design presented in the previous section is meant to address all identified
requirements, only preliminary results for responsiveness and fairness is presented in this section.

\subsection{Simulation model}
\label{test_setup}

%Implementation
The proposed multi-tiered model is currently being implemented in Oversim \cite{OverSim_2007}, a P2P simulation environment based in Omnet++, which
allows for the measurement of identified requirements. Furthermore, it allows for the comparison of the current model with other state persistency
models. Initial results are promising, with the implemented model functioning as expected. The simulated system is very responsive when storing data
within a group and as responsive as storing data in an overlay, when storing data between groups.

Pithos is designed to form part of a complete P2P MMVE network solution. It is assumed that there exists some intelligence that drives Pithos and has
to determine when objects have to be stored and retrieved on each peer. This driving intelligence is termed the game layer.

In the results shown, Pastry was used as the P2P overlay and Pithos was driven by a \emph{Game} module developed for this purpose. After a node has
joined a group, the game module starts to generate store and retrieve requests at a rate of 10 objects per second and a size of 1024 bytes. The size
of 1024 byte objects was chosen to be much larger than Quake 3 game objects without delta encoding, used in \cite{Bharambe_Donnybrook}. Pithos is
designed for the low latency storage of small game objects.

For the results shown, 14499 peers, 500 super peers and a single directory server are created at the start of the simulation. The directory server
publishes super peer information, which allows a peer to join the group nearest to it. Because of the way Pithos is structured, each super peer node
is also a peer node, which gives a total of 15000 Oversim nodes.

To be able to simulate Pithos for 15000 nodes, it runs on the Oversim simple underlay network \cite{oversim_applications}, where node latencies are
determined by the distance between nodes placed in an $n$-dimensional Euclidean space. The positions of the nodes are chosen to match the latencies
of the CAIDA/Skitter project. Different nodes are also assigned different bandwidth and jitter parameters to simulate a heterogenous network.

\subsection{Responsiveness}

To exactly compare Pithos with overlay storage, the probability that a message is routed within a group ($P(g)$) should first be known. It is
expected that $P(g)$ will be different for MMVEs with different defining mechanics. It should be possible to determine $P(g)$ experimentally for a
specific type of game, but this will require access to the game client of an already implemented P2P MMVE. The exact measurement of $P(g)$ is left
for future work, but the responsiveness can be calculated as a function of $P(g)$. Working with a function in $P(g)$ also allows for the
implementation of various dynamic strategies that can adapt to various values of $P(g)$.

We define $P(o) = 1 - P(g)$ as the probability that a message is routed within the overlay. We define $T_{\textrm{group}}$ as the root and replica
message distribution and $T_{\textrm{overlay}}$ as the overlay message distribution, both shown in Figure \ref{fig_pithos_response}. The expected
value of the overall system response time ($E[T_{\textrm{resp}}]$) can then be presented as a weighted average of the expected values of both the
group and overlay storage distributions, as follows:
%
\begin{align}
    E[T_{\textrm{resp}}] &= P(g)\left(E\left[T_{\textrm{group}}\right]\right) + P(o)\left(E\left[T_{\textrm{overlay}}\right]\right)\notag\\
                         &= P(g)\left(E\left[T_{\textrm{group}}\right]\right) + \left[1 - P(g)\right]\left(E\left[T_{\textrm{overlay}}\right]\right).\label{expected_response_time}
\end{align}

\begin{figure}[htbp]
 \centering
 \includegraphics[clip=true, viewport=1cm 0.5cm 29cm 20.5cm, width=\columnwidth]{StoreTimes}
 \caption{(top) Time distribution of overlay and root/replica objects, (bottom) time distribution of Pastry objects.}
 \label{fig_pithos_response}
\end{figure}
%
Figure \ref{fig_pithos_response} (top) shows the distribution of mean storage request times over all nodes in the Pithos network for the different
storage types. One can see that the intra-group root and replica objects are stored much faster ($E\left[T_{\textrm{group}}\right] = 0.0878s$) than
the overlay objects in the network ($E\left[T_{\textrm{overlay}}\right] = 0.328s$). Figure \ref{fig_pithos_response} (bottom) presents the
responsiveness of a pure Pastry network of 14999 nodes in Oversim. The simulated Pastry network was found to have a mean routing time of $0.12s$.

The responsiveness of Pithos will depend on the responsiveness of Pastry, where the expected number of Pastry hops are given by:
\cite{storage_and_chaching_PAST}:
%
\begin{equation}\label{pastry_hops}
    E[H_{\textrm{pastry}}] = \log_{2^b}\left(N\right),
\end{equation}
%
where $b$ is a network parameter that is usually chosen as $b = 4$. From this, it is possible to calculate a theoretical performance for Pithos and
compare that with a theoretical performance of overlay storage.

When using a weighted hop average, as with Equation \eqref{expected_response_time}, the expected number of Pithos hops is given by:
%
\begin{equation}\label{expected_response_time}
    E[H_{\textrm{pithos}}] = P(g)\left(E\left[H_{\textrm{group}}\right]\right) + P(o)\left(E\left[H_{\textrm{overlay}}\right]\right),
\end{equation}
%
where $E\left[H_{\textrm{group}}\right]$ is the expected number of group hops and $E\left[H_{\textrm{overlay}}\right]$ is the expected number of
overly hops. In Pithos, $E\left[H_{\textrm{group}}\right] = 1$, because in a fully connected group any node is always one hop away from any other
node.

To find the value of $E\left[H_{\textrm{overlay}}\right]$, one has to consider how many hops an overlay message requires in Pithos. One hop is
required to send a store request from a group peer to its super peer. The super peer then forwards the message to another super peer in
$\log_{16}(M)$ hops, from Equation \eqref{pastry_hops}, where $M$ is the number of super peers in the network. From the destination super peer,
another hop is required to send the message to the destination group peer. This gives:
%
\begin{equation}\label{group_hops}
    E\left[H_{\textrm{overlay}}\right] = 1 + \log_{2^b}(M) + 1.
\end{equation}
%
Equation \eqref{expected_response_time} then becomes:
%
\begin{align}
E[H_{\textrm{pithos}}] &= P(g) + \left[1 - P(g)\right]\left[2 + \log_{16}\left(M\right)\right]\notag\\
                       &= 1 + \left[1 - P(g)\right] + \left[1 - P(g)\right]\left[\log_{16}(M)\right]\notag\\
                       &= 1 + \left[1 - P(g)\right]\left[1 + \log_{16}(M)\right]\notag\\
                       &= 1 + P(o)\left[1 + \log_{16}(M)\right].\label{expected_response_time_exp}
\end{align}

\begin{figure}[htbp]
 \centering
 \includegraphics[clip=true, viewport=0cm 5cm 27cm 14.5cm, width=\columnwidth]{Hops_vsGroupFrac_4n}
 \caption{Expected number of Pithos hops, compared to the expected number of overlay hops, as a function of $P(g)$ for various values of $N$}
 \label{fig_hop_compare}
\end{figure}
%
Figure \ref{fig_hop_compare} compares the expected number of Pithos hops with the expected number of overlay hops as a function of intra-group
probability ($P(g)$) for various numbers of nodes ($N$). The overlay hops were calculated from Equation \eqref{pastry_hops}, while the Pithos hops
were calculated from Equation \eqref{expected_response_time_exp}. For the Pithos graphs, an average number of 50 peers per group was used to
determine the number of super peers.

Figure \ref{fig_hop_compare} shows that for a low value of $P(g)$, overlay storage performs better than Pithos because of the additional two hops
present in Pithos. High values for $P(g)$ are expected, because of the distance-based design of Pithos that attempts to maximise the value of $P(g)$.
This should have Pithos perform better than overlay storage.

\subsection{Fairness}

\begin{figure}[htbp]
 \centering
 \includegraphics[clip=true, viewport=1cm 0.5cm 28.5cm 20cm, width=\columnwidth]{RootRepOverlayObjects}
 \caption{(top) Root/Replica object number distribution, (bottom) overlay number distribution.}
 \label{fig_group_overlay_objects}
\end{figure}
%
To evaluate the fairness, we evaluate the standard deviation of the number of objects stored per peer. Figure \ref{fig_group_overlay_objects} (top)
shows the distribution of group objects over nodes in the network. The figure shows how many nodes store how many objects. The distribution has a
mean and standard deviation of 302 and 51 objects per node respectively.

Figure \ref{fig_group_overlay_objects} (bottom) shows the distribution of overlay objects in Pithos with a mean and standard deviation of 153 and 189
objects per node respectively. Comparing the standard deviations of group storage to overlay storage, it appears that group storage is much fairer
than overlay storage. This shows that by designing a hybrid system which prefers group storage to overlay storage, one is also designing a fairer
system than overlay storage.

\begin{figure}[htbp]
 \centering
 \includegraphics[clip=true, viewport=1cm 5cm 29cm 14.5cm, width=\columnwidth]{Objects}
 \caption{Combined object number distribution}
 \label{fig_objects}
\end{figure}
%
Figure \ref{fig_objects} shows the combined object distribution of Pithos, with a mean and standard deviation of 453 and 200 objects per node
respectively. This shows that the fairness of Pithos is currently dominated by the fairness of Pastry and that Pithos is as fair as overlay storage.

\section{Conclusion}
\label{conclusion}

\subsection{Summary}

In this paper, it was found that none of the storage types reviewed satisfied all identified requirements of P2P MMVE storage systems. A novel
storage architecture called Pithos was presented to satisfy all the identified requirements. The key characteristics of Pithos is that is groups
peers and thereby forms a two-tiered storage architecture to promote low latency interactions amongst group peers. It employs storage replication to
ensure reliable storage. It uses distance-based storage to determine on which groups objects will be stored. This is to ensure that objects stored
within a group are of interest to the group. Pithos also makes use of secure ID assignments and request signing to promote storage security. For
preliminary results, the requirements of responsiveness and reliability were compared to those of overlay storage and Pithos was found to be more
responsive than overlay storage and as fair.

\subsection{Future work}
\label{future_work}

%Complete implementation
We plan to complete the Pithos architecture simulation and compare all requirements with all identified storage types. The simulation does not yet
support network churn. A migration mechanism should still be implemented before testing under churn can be done. Another improvement is adding all
peers to the storage overlay in situations where there are relatively few peers per group. This will improve responsiveness by removing the two extra
hops present in Equation .

A key aspect of completing Pithos will be how players are grouped. Future research will focus on user behaviour, clustering algorithms and dynamic
regioning approaches. The different clustering techniques should be compared and the most applicable one will be chosen and improved to drive Pithos.

%Driver data
A model of data storage and retrieval requests is also being developed. This includes sizes of objects stored, how regularly these objects are stored
and what latency requirements exist for object retrieval. It is assumed that these values will depend on the specific MMVE and therefore different
storage parameters should also be identified.


%\newpage
% use section* for acknowledgement
\ifCLASSOPTIONcompsoc
  % The Computer Society usually uses the plural form
  \section*{Acknowledgments}
\else
  % regular IEEE prefers the singular form
  \section*{Acknowledgment}
\fi

The financial assistance of MIH and the National Research Foundation (NRF) towards this research is hereby acknowledged. Opinions expressed and
conclusions arrived at, are those of the author and are not necessarily to be attributed to MIH or the NRF.

%\newpage
%\IEEEtriggeratref{43} %Balance the bibliography
\bibliographystyle{IEEEtran}
\bibliography{../BibTeX/P2P_MMOG}

% that's all folks
\end{document}
