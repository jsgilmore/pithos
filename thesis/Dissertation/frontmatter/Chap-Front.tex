
\chapter*{Abstract}
Stellenbosch University and the Katholieke Universiteit Leuven has a joint undertaking to develop
a satellite communications payload. The goals of the project are: to undertake research
and expand knowledge in the area of dynamically configurable antenna beam forming, to prove
the viability of this research for space purposes and to demonstrate the feasibility of the
development in a practical application.

The practical application is low Earth orbit satellite communication system for applications in remote monitoring.
Sensor data will be uploaded to the satellite, stored and forwarded to a central processing
ground station as the satellite passes over these ground stations. The system will utilise many
low-cost ground sensor stations to collect data and distribute it to high-end ground stations
for processing.

Applications of remote monitoring systems are maritime- and climate change monitoring-
and tracking. Climate change monitoring allows inter alia, for the monitoring of the effects and causes
of global warming.

The Katholieke Universiteit Leuven is developing a steerable antenna to be mounted on the
satellite. Stellenbosch University is developing the communications payload to steer and use
the antenna. The development of the communications protocol stack is part of the project.
The focus of this work is to implement the application layer protocol, which handles all file level
communications and also implements the communications strategy.

The application layer protocol is called the \emph{Satellite Communications Software System}
(SCSS). It handles all high level requests from ground stations, including requests to store
data, download data, download log files and upload configuration information. The design
is based on a client-server model, with a \emph{Station Server} and \emph{Station Handler}.
The Station Server schedules ground stations for communication and creates a Station Handler
for each ground station to handle all ground station requests. During the design, all file
formats were defined for efficient ground station-satellite communications and system administration.
All valid ground station requests and handler responses were also defined.

It was also found that the system may be made more efficient by scheduling ground stations
for communications, rather than polling each ground station until one responds. To be able
to schedule ground station communications, the times when ground stations will come into
view of the satellite have to be predicted. This is done by calculating the positions of the
Satellite and ground stations as functions of time. A simple orbit propagator was developed to
predict the satellite distance and to ease testing and integration with the communications system.
The times when a ground station will be within range of the satellite were then predicted and
a scheduling algorithm developed to minimise the number of ground stations not
able to communicate.

All systems were implemented and tested. The SCSS executing on the Satellite was
developed and tested on the satellite on-board computer. Embedded implementations possess
strict resource limitations, which were taken into account during the development process.
The SCSS is a multi-threaded system that makes use of thread cancellation to improve
responsiveness.


\chapter*{Samevatting}
\hyphenation{grond-sta-sies}

Die Universiteit van Stellenbosch ontwerp tans 'n satelliet kommunikasieloonvrag in
samewerking met die Katolieke Universiteit van Leuven. Die doel van die projek is om
navorsing te doen oor die lewensvatbaarheid van dinamies verstelbare antenna bundelvorming
vir ruimte toepassings, asook om die haalbaarheid van hierdie navorsing in die praktyk
te demonstreer.

Die praktiese toepassing is 'n satellietkommunikasiestelsel vir afstandsmonitering,
wat in 'n Lae-Aarde wentelbaan verkeer. Soos die satelliet in sy wentelbaan beweeg,
sal sensor data na die satelliet toe gestuur, gestoor en weer aangestuur word. Die
stelsel gebruik goedkoop sensorgrondstasies om data te versamel en aan te stuur na
kragtiger grondstasies vir verwerking.

Afstandsmoniteringstelsels kan gebruik word om klimaatsverandering, sowel as die
posisie van skepe en voertuie, te monitor. Deur oa. klimaatsveranderinge te dokumenteer,
kan gevolge en oorsake van globale verhitting gemonitor word.

Die Katholieke Universiteit van Leuven is verantwoordelik vir die
ontwerp en vervaardiging van die satelliet antenna, terwyl die Universiteit van
Stellenbosch verantwoordelik is vir die ontwerp en bou van die
kommunikasie loonvrag. 'n Gedeelte van hierdie ontwikkeling sluit die
ontwerp en implementasie van al die protokolle van die
kommunikasieprotokolstapel in. Dit fokus op die toepassingsvlak
protokol van die protokolstapel, wat alle le\^{e}rvlak kommunikasie
hanteer en die kommunikasiestrategie implementeer.

Die toepassingsvlaksagteware word die Satellietkommunikasie sagtewarestelsel
(SKSS) genoem. Die SKSS is daarvoor verantwoordelik om alle navrae
vanaf grondstasies te hanteer. Hierdie navrae sluit die
oplaai en stoor van data, die aflaai van data, die aflaai van logs en
die oplaai van konfigurasie inligting in. Die ontwerp is op die standaard
kli\"{e}nt-bediener model gebasseer, met 'n \emph{stasiebediener} en 'n
\emph{stasiehanteerder}. Die stasiebediener skeduleer die tye wanneer
grondstasies toegelaat sal word om te kommunikeer en skep stasiehanteerders om alle
navrae vanaf die stasies te hanteer. Gedurende die ontwerp is alle
le\^{e}rformate gedefinieer om doeltreffende adminstrasie van die
stelsel, asook kommunikasie tussen grondstasies en die satelliet te
ondersteun. Alle geldige boodskappe tussen die satelliet en grondstasies
is ook gedefnieer.

Daar is gevind dat die doeltreffendheid van die stelsel verhoog kan word deur die
grondstasies wat wil kommunikeer te skeduleer, eerder as om alle stasies
te pols totdat een reageer. Om so 'n skedule op te stel, moet die tye
wanneer grondstasies binne bereik van die satelliet gaan wees voorspel
word. Hierdie voorspelling is gedoen deur die posisies van die
satelliet en die grondstasies as funksies van tyd te voorspel. 'n
Eenvoudige satelliet posisievoorspeller is ontwikkel om toetsing en
integrasie met die SKSS te vergemaklik. 'n Skeduleringsalgoritme is toe
ontwikkel om die hoeveelheid grondstasies wat nie toegelaat word om te
kommunikeer nie, te minimeer.

Alle stelsels is geimplementeer en getoets. Die SKSS, wat op die
satelliet loop, is ontwikkel en getoets op die satelliet se aanboord
rekenaar. Die feit dat ingebedde stelsels oor baie min hulpbronne beskik,
is in aanmerking geneem gedurende die ontwikkeling en implementasie van die SKSS.
Angesien die SKSS 'n multidraadverwerkingsstelsel is, word daar van
draadkansellasie gebruik gemaak om die stelsel se reaksietyd te verbeter.


\chapter{Acknowledgements}%==================================================

I would like to express my sincere gratitude to the following people and organisations:
\begin{itemize}
  \item the Holy Father, for keeping me and blessing me with so much;
  \item my study leader, Dr Riaan Wolhuter, for his continued guidance and support;
  \item my fianc\'{e}e, Jacki van der Merwe, for her lasting love, support and understanding;
  \item Francois Olivier and Shaun Lodder, for their valuable input during the late nights in the lab;
  \item Dr Gert-Jan van Rooyen for his valuable feedback on the SCSS design;
  \item Ewald van der Westhuizen for managing the Leuven project and for providing technical assistance;
  \item Kobus Botha for always being ready to assist with technical issues;
  \item Japie Engelbrecht, for helping me better understand satellite communication systems;
  \item the Telkom Centre of Excellence and Stellenbosch University, for their financial aid;
  \item my parents, John and Coreen Gilmore, for making me the man I am today and making
  my studies possible;
  \item the QNX support team, for their prompt and knowledgeable assistance with QNX related implementation issues;
  \item James Clark, for writing the Expat XML parser library;
  \item Jean-Loup Gailly and Mark Adler, for writing the zlib compression library.
\end{itemize}


\chapter{Dedications}%=======================================================
 \vfill
 \begin{center}\itshape
    In memory of my mother, Anita Gilmore, and my grandparents: Herman Kotze, Kotie Kotze and Hettie Gilmore.
	I hope I've made you proud.
 \end{center}
 \vfill
 \clearpage

%============================================================================
\endinput
