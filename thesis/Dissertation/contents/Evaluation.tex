\chapter{Pithos Evaluation}
    \label{chp:EVALUATION}

In the results shown, Pastry was used as the P2P overlay and Pithos was driven by a \emph{Game} module developed for this purpose. After a node has
joined a group, the game module starts to generate store and retrieve requests at a rate of 10 objects per second and a size of 1024 bytes. The size
of 1024 byte objects was chosen to be much larger than Quake 3 game objects without delta encoding, used in \cite{Bharambe_Donnybrook}. Pithos is
designed for the low latency storage of small game objects.

For the results shown, 14499 peers, 500 super peers and a single directory server are created at the start of the simulation. The directory server
publishes super peer information, which allows a peer to join the group nearest to it. Because of the way Pithos is structured, each super peer node is also a peer node, which gives a total of 15000 Oversim nodes.

    %%%%%%%%%%%%%%%%%%%%%%%%%%%%%%%%%%%%%%%%%%%%%%%%%%%%%%%%%%%%%%%%%%%%%%%
    \section{Performance}

    \subsection{Bandwidth requirements (overhead)}
    %Mention something about low bandwidth links and how that influences timeout and what extra mechanisms were required.
    %Check how this increases with increased network sizes and varying levels of network churn.

        \subsection{Group consistency}
        %Check how this increases with increased network sizes and varying levels of network churn.

    \section{Metrics}

    \section{PithosTestApp}

    \section{Comparison between Pithos and other architectures}

        \subsection{Overhead}
        \subsection{Fairness}
        \subsection{Reliability}
        \subsection{Responsiveness}
