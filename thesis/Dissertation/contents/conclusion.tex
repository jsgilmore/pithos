\chapter{Conclusions and Recommendations}
\label{chp:CONC}


%%%%%%%%%%%%%%%%%%%%%%%%%%%%%%%%%%%%%%%%%%%%%%%%%%%%%%%%%%%%%%%%%%%%%%%
%\section{Introduction}

%The final chapter in this work presents the conclusions drawn from the work performed as well
%as making some recommendations on further work that might be performed. Two main areas
%are investigated in this work. The one is the design of the communications strategy with the
%focus on the link acquisition protocol and the other is the design and implementation of the
%SCSS, driven by the designed communications strategy.

%Section \ref{comms_strategy} discusses the communications strategy used.
%
%Section \ref{software_system} discusses the SCSS design.
%
%Section \ref{contributions} describes contributions made by this work.
%
%Section \ref{further_work} presents some areas where there further work can be undertaken
%to improve the link quality prediction, communications strategy and the SCSS.

\section{Communication strategy}
\label{comms_strategy}

When designing the SCSS, various methods were investigated for a means to efficiently establish
the satellite-ground station connection. A few major factors specified the design of the protocol.
These were:
%
\begin{enumerate}
\item the satellite mounted steerable antenna,
\item the need to produce low cost ground stations and
\item the single available communications channel on the satellite.
\end{enumerate}

With the satellite mounted steerable antenna, a sufficient link margin was achieved, to not require steerable
antennas on the ground stations. This did, however, create the requirement that the satellite, and not the
ground stations, should initiate connections, as described in Section \ref{tracking_link_acq}. Two methods
of tracking were also investigated: program tracking and tracking techniques, making use of satellite beacon
transmissions, as presented in Section \ref{leo_comms}. Program tracking was chosen, as this requires
no transmission of beacon packets, which frees up the communications channel for data.

A broadcast method may be used for a single channel, where the satellite broadcasts until a
ground station connection has been established, processes all ground stations requests, and
then broadcasts again to establish a new connection. This method is inefficient, as no knowledge
of the link quality or available communications time is taken into account. 

If the system possesses knowledge of link quality and communications duration, it can intelligently
schedule ground station connections, to optimise the volumetric data throughput of the system.
Since the satellite has a known orbit track and the ground station also moves predictably with
the movement of the Earth, the distance between these two entities may be predicted. The
distance prediction produces a measure of link quality and, because time can also be linked
to the movements, the communication duration may also be predicted.

The predictions can then be used to schedule ground stations in such a way as to maximise
the total communication time of the system and minimise the number of ground stations, not
allowed to communicate. The schedule can then be uploaded to the satellite, from where
connections are made. When the schedule is created, the total amount of communications time
is also equalised among ground stations, which produces a fair schedule. These two techniques
produce a more efficient method of link acquisition, than a system with no past or future knowledge
of the established links.

In scenarios where the link margin is not sufficient for communications, directional ground station
antennas may be used. These antennas have to be pointed in some specific direction and only
when the satellite flies through the antenna beam, will it be able to communicate. Angle predictions
were performed to aid in calculating the optimal direction, in which the antenna should point, to
communicate with the satellite for the maximum amount of time.

\section{Satellite Communications Software System}
\label{software_system}

With the communications strategy devised, the SCSS was designed. The design is
based on the client-server model of communications, where the server runs and creates handler
threads for clients requiring service. From this model, the schedule was incorporated into
the server.

A deviation from the classic client-server model is present in the connection setup phase 
of the SCSS. In the classic model, clients initiate connections to the server, but
because of the schedule controlled communication times, the server must initiate connections
to ground stations. This difference does not, however, change the basic functionality of the
SCSS from that of a classic client server model.

Multi-threading also allowed for the modularisation of the the SCSS. This simplified development,
decreased the level of coupling and promoted encapsulation of data and functions. The multi-threaded
approach also improved the robustness of the SCSS. If an unexpected scenario occurs, from
which the SCSS cannot correct itself, only the current ground station connection is lost and
further ground station may still be handled by other station handlers.

During development, resource limitations were identified as a potential development risk. To
manage this risk, various mechanisms were implemented to reduce memory and CPU usage.
This led to an efficient design, well suited for an embedded implementation.

The designed SCSS performed well under testing and well within resource limits
set by the hardware. The thread cancellation mechanism used, requires involved memory
management techniques, but the mechanism also increases the responsiveness of the SCSS.
The cancellation mechanism ensures that no ground station can use more than its assigned
communications time.

The schedule driven design proved to be a robust and efficient design, while implementing
all features required by other sub-systems and ground stations. Features required by other
sub-systems include providing a pointing target for the antenna control software to point to
and providing the lower lever protocol layers with data to transmit. Features required by
ground stations are those enabling the ground stations to implement a store-and-forward
system over the satellite link for remote monitoring applications. These features include:
allowing ground stations to communicate with the satellite on a file level, allowing ground
stations to upload and download data and allowing ground stations to send data to other
ground stations.

\section{Contributions}
\label{contributions}

The three main contributions of this work, are:
\begin{enumerate}
\item the development of a communications strategy,
\item design and implementation of the SCSS
\item and a centralised control approach.
\end{enumerate}

The main contributions of the SCSS are:
\begin{itemize}
\item A resource efficient and multi-threaded high level communications control system
with a high level of responsiveness.
\item The scheduler, which determines the communication times of ground stations.
\item The definition of all messages and their formats, which will be used for satellite-
ground station communications.
\item The Station Server commanding the antenna control software and thereby
supporting the satellite mounted steerable antenna.
\end{itemize}

The main contributions of the communications strategy are:
\begin{itemize}
\item The possible maximisation of volumetric data throughput of the satellite system.
\item This is achieved by efficiently scheduling ground station communications.
\item The calculation of the satellite position as a function of time, using orbital elements and
orbital mechanics, enable the calculation of the communications schedule.
\item Fair assignment of communication times are also ensured by the scheduling algorithm.
\item Position prediction also allows for the calculation of the communications statistics
by making use of visibility predictions. These predictions were compared to the STK
software package and found to match closely with the values predicted by the STK.
\item Predicted total communication times enable system designers to calculate the
overall system capacity as well as the capacity of each ground station, according to its geographic location.
\item Angle predictions also enable system designers to optimally point static directional ground
station antennas. Directional antennas may be required in areas where the link margin is
insufficient. The angle predictions provide an optimal elevation angle, as well as an optimal
horizontal angle, measured from true North.
\end{itemize}

The main contributions of the implemented centralised control are:
\begin{itemize}
\item The promotion of on-board processing techniques as well as satellite autonomy.
\item Ease of administration and coordination.
\item Improved response time because of the link acquisition strategy as explained in Section \ref{on-board_proccessing}.
\item By using satellite controlled link acquisition, together with a satellite mounted steerable antenna,
steerable antennas can be removed from ground stations. This drastically decreases the
costs of ground stations.
\end{itemize}


\section{Further work}
\label{further_work}

%Perform integration testing between software system and rest of protocol stack.
Firstly, the SCSS should be integrated with the rest of the communications protocol
stack. This could not be done during the implementation of the SCSS, as the other
projects were not at a point in their development where they could be integrated. After integration,
integration testing should be performed on the communications system as a whole.
This should include sending a file over an RF link and checking whether the file was received.
Smaller tests between the SCSS and the antenna control software should also be performed.

Further required testing includes protocol tests, protocol efficiency evaluation and a comparison
of these evaluations against other communications protocols. Protocol efficiency evaluation
includes calculating the protocol overhead during a typical transmissions.

%Directly download TLE's from Spacetrack
For the position prediction scheme, the schedule generator should be automated to download
the two-line-elements of the satellite from Spacetrack. This will enable the schedule generator to
always use up-to-date orbital elements to achieve a maximum accuracy of orbit prediction.

%Use commercial SGP4 algorithm, instead of custom algorithm
For the orbital prediction itself, a more accurate orbit propagator can be used, for example the SGP4 algorithm
or an improved variant thereof. This will also improve the accuracy of the visibility prediction and take
into account orbit perturbations. A more accurate orbit propagator requires less frequent two-line-element
updates. The corollary of this is that a less accurate orbit propagator may be used, when
two-line-elements are updated more frequently.

%Add and compare more scheduling schemes
For the schedule generator, the performance of more scheduling schemes should be compared.
Different schemes should be investigated to determine whether other schemes exist, that further
reduce the number of ground stations not able to communicate. The \emph{least number of exclusions} (LNE)
scheme, proposed in this work, should also be compared to other schemes. The LNE scheme can
also be further extended by making the scheme $n$ levels deep, instead of one.

Currently, the scheduler looks ahead one schedule time, to determine how many stations will
be excluded. But this scheme can be extended, to evaluate all possibilities and find the scheme
with the total minimum number of exclusions. This scheme will most likely have a very high order
complexity.

It could also be of worth further investigating the scheduling problem from a purely mathematical
perspective. An optimal scheme may be proposed for the defined scheduling problem.

Another algorithm should be developed to optimise the overall communications system, when
directed antennas are taken into account. The predicted angle positions of every ground station
in the system should be predicted and overlaid, with the CTWs. After all CTWs are scheduled,
the optimal angle should be used from every ground station. This angle would preclude other
possible communications times for that ground station. The scheduling algorithm will have to
be rerun. Times when a scheduled ground station could communicate, may now no longer be
valid. This will give rise to different angles that should be optimised. An algorithm should be
developed, enabling this calculation to converge to a system where the total amount of
communications time is maximised.

Quality of service can also be implemented in the SCSS if more finely grained
priority is used. The priority can then be used as a weight during the scheduling algorithm,
together with the total allocated communications time, to create a schedule where some ground
station receive proportionally more communications time than others.

%\section{Summary}

%The final chapter in this work, presented the important conclusions, contributions and further
%work of the this thesis.

%Initially, the chapter re-examined the two main parts of this thesis: the SCSS and the
%communications strategy. A short overview was given of each of these areas
%and a summarised view was presented of the development of each of the two areas in the work.

%The contributions of this work were then summarised by again reviewing the main focus
%areas of this work. The contributions include the design and implementation of the SCSS,
%the development of the communications strategy and the position prediction and schedule
%generation that go with it, as well as the move to a on-board processing satellite with aspects
%of autonomy.

%The chapter concluded with further work that might be undertaken in each of the research areas
%touched.  The section again reviewed all relevant areas and discussed ways of further improving
%the satellite system as a whole.