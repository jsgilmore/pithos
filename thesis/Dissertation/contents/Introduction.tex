\chapter{Introduction}
\label{chp:INTRO}


%%%%%%%%%%%%%%%%%%%%%%%%%%%%%%%%%%%%%%%%%%%%%%%%%%%%%%%%%%%%%%%%%%%%%%%
\section{Background}
\label{background}

The IS-HSII project is a joint project undertaken between the Katholieke Universiteit (KU) Leuven and Stellenbosch
University for the development of a satellite borne, electronically beam steerable antenna array. The
ESAT-TELEMIC division of the Department of Electrical Engineering, of the KU Leuven, is currently developing
techniques for electronic antenna beam steering in space. The purpose of the development is threefold:
\begin{itemize}
\item To undertake research and expand knowledge in the area of dynamically configurable antenna beam
forming as an academic objective
\item To prove the viability of this research for space purposes
\item To demonstrate the feasibility of the development in a practical application, where ground-based
environmental sensor data would be uploaded to a satellite carrying the steerable antenna array
\end{itemize}

The project is jointly undertaken between the KUL and the Digital Signal Processing (DSP)-Telecommunications group of the Department
of Electrical and Electronic Engineering of Stellenbosch University. The antenna and
associated components are developed in Leuven, while the satellite platform, ground station and
ground-satellite communications link, are developed in Stellenbosch. The eventual objective is to fly the
system on the next South African satellite.

Part of the ground station-satellite communications link is the design of the protocol stack of the
communications sub-system and the implementation of all layers thereof, to enable full store-and-forward functionality.
A team of people were appointed to implement the communications sub-system.

The implementation consists of both software and hardware development. Most of the hardware was developed
by Sunspace for the SumbandilaSat project. The on-board computer used for the project is the same
computer as on SumbandilaSat. Some experience could be transferred from the SumbandilaSat project to the Leuven
project, but there were also some major differences. A key difference is the Sumbandila satellite is
a half-duplex system, whereas the Leuven system is full-duplex. This allows for more freedom and functionality
in the on-board software design.

One component of the communications system is the Satellite Communications Software System (SCSS) that
resides in the top layer of the protocol stack, in the application layer. The SCSS controls communication times
and durations with ground stations, stores files received from ground stations and implements a store-and-forward system
to deliver data to destination ground stations. The design and implementation of the SCSS is the focus
of this work.

\section{Objectives and contributions}
\label{objectives}

The objectives of this study was the design and implementation of the SCSS executing on the satellite
on-board computer, while taking into account the resource limitations of the hardware. The design
includes defining all message formats to be used for communications between the SCSS and ground stations.
The purpose of the SCSS is to coordinate all high level communication operations of the satellite.

Functionality expected from the SCSS are:
\begin{itemize}
\item Initiate connections with ground stations
\item Allow ground stations to communicate with the satellite on a file level
\item Allow ground stations to upload and download data
\item Allow ground stations to send data to other ground stations
\item Manage the communication times and durations to ensure that all ground stations receive equal service.
\item Provide a means to store and retrieve uploaded data on the satellite
\item Manage the steering of the satellite antenna, to point to the currently communicating ground station.
\end{itemize}

The SCSS was successfully developed and implemented on the satellite hardware. The SCSS
consists of a station scheduler that manages the allowed communication times of all ground stations and
the station handlers, which directly handle all requests from ground stations. A file store is also implemented
to store all ground station messages, configuration data and schedules. Testing was performed to ensure the
SCSS functions as designed and within the required resource constraints.

A substantial amount of time was spent on developing an effective communications strategy. The strategy
manages communication times with ground stations, including both allowed start times and communications
duration. It was found that the volumetric throughput of the system could be maximised, by actively controlling
ground station communication times. The active control is performed by the satellite and involves a schedule
calculated off-line, making use of satellite and ground station position predictions.

Predicting the satellite and ground station positions, allows the satellite to be aware of the link quality
of every satellite pass. This allows a scheduler to select a ground station, having a high-quality link for
the specific pass. In this way, every ground station is allowed a high quality pass when it communicates,
while the strategy also equalises all ground station communication times. The calculated schedule drives
the SCSS.

The link predictions also allows for the analysis of the space mission, to calculate average and total communication
times and thereby the overall system communications capacity.

Angle predictions were also performed to allow for a directed antenna on the ground stations, instead of
a basic omni-directional design. Angle predictions allow for the average angle to be predicted, which
the satellite and ground station will communicate in, most of the time. This can be used to achieve
an acceptable link margin, in applications where an omni-directional antenna produces an unacceptable
link margin.

\section{Applications}
\label{applications}

The satellite system under discussion is being designed for remote monitoring applications, as stated in Section
\ref{background} and depicted in Figure \ref{fig_comms_overview}. A remote monitoring system exists of two types
of ground stations, i.e. ground stations collecting data from sensor networks and those aggregating the collected
data for further processing, or pass the data on to a server in the Internet for processing. The ground stations
collecting the sensor data, are the data sources in the communications system and the ground stations aggregating
the data, are the data sinks in the system.

Ground stations collecting sensor data are placed in remote areas, with no Internet or cellular connection.
There is, therefore, no way for data collected from these sensors to be processed without manual data collection
techniques. This manual process of data collection is very labour intensive and allows little time for data processing
during collection, or requires a large team to collect the data.

A remote monitoring system collects the data from rural ground stations and only requires personnel to process the
data at the data processing centre where the data are sent. Data are uploaded from rural ground stations and downloaded
to the aggregator ground station where all data are stored in a database, or sent to a server for processing. Data mining
can then be performed from a central location and different data sets from different sensor ground stations can easily
be correlated and compared.

Applications of remote monitoring systems are tracking, and climate and maritime monitoring. Climate monitoring
systems monitor meteorological elements such as temperature, humidity, rainfall, wind and atmospheric pressure
in a given region. In current times, these systems are of  great importance, because they enable the monitoring
of the effects of global warming. It allows tracking of the global climate at near to real-time. Large data sets can
then be processed after being delivered to a central processing facility with more powerful capabilities than what
would be individually available at every monitoring site.

The second example is that of tracking. Environmental research is done at the University of Stellenbosch
to track Leopard movements with tracking collars. Another application of the satellite system is to
have ground sensor stations monitoring the positions of the tracking collars. Multiple ground stations
can then be set up to monitor leopards moving throughout their habitat. These stations will be set up
in rural areas in the mountains and will have no form of communications, except the satellite system.
Manual data retrieval is also difficult in these areas as researchers have to track the leopards with
tracking equipment in the field. This can be a tedious as well as dangerous expedition.

The final example is one where a satellite based tracking antenna can have a significant impact. This
is in maritime monitoring operations. Shipping companies monitor their ships to enable them to gauge their times
to arrival as well as other operational parameters. Currently, these ships have systems to communicate
with a satellite and these communications are made possible with the use of an antenna enabling contact
with the satellite. It is important that the antenna be pointed at the satellite at all times when communicating.
The antenna must, therefore, possess stabilisers that maintain the correct pointing direction, even with
the destabilising effect of ocean waves.

Antennas currently used on ships are very expensive, as they employ sophisticated stabilisation techniques
\cite{stable_antenna}. With the antenna mounted on the satellite, there is no need for stabilisation on
the ship. The satellite is always able to point towards the ship and the movement of the ship will have no
significant effect, provided that the ship-mounted antenna is reasonably omni-directional.

For the design of the satellite communications system, the intended applications should at all times be kept
in mind. The characteristics of a remote monitoring systems are low data volumes, high number of
data sources, intermittent data production, text-based measurement data that are highly compressible
and data not requiring immediate processing or transmission.

The low data volumes stem from the nature of the data. The data are measurement data, which would
consist of numbers and text. No video or audio data are transmitted. Since there may be many sensor networks
and rural ground stations throughout the area being monitored, many data sources using the satellite exist in
the system. The sensor system will, however, not be constantly producing data. Measurements are
taken at certain times during the day and that data must then be uploaded for aggregation. The measurement
data are also considered to be non-real-time off-line data. No immediate data processing is required
and a delay in transmission will not adversely affect the system.

\section{Overview of this work}
\label{overview_intro}

Chapter \ref{chp:LIT} provides the required background information on satellite communications and how
this applies to LEO satellite systems and specifically to the LEO satellite system being designed.
The chapter takes a top-down approach to provide perspective on where the SCSS fits in.
It concludes by describing on-board processing and satellite autonomy, which forms part of the
focus of this work.

After satellite systems in general have been described, Chapter \ref{chp:system_overview} presents an
overview of specifically the satellite being designed. It details the orbit characteristics to show what
little time is available for LEO communications. It also describes the basics of how the satellite and
the ground station will communicate and presents an overview of the protocol stack developed. The
chapter then moves on to describe the hardware and interfaces of the system as well as to give
some background information on the architecture of the operating system used on the on-board computer.
The chapter also describes the KU Leuven steerable antenna, around which the project was designed and
discusses how the antenna influenced design.

Chapter \ref{chp:LINK_ACQ} describes a novel method of how the first function of the SCSS
is implemented, namely, acquiring the satellite link. The chapter introduces the link acquisition technique
and shows how links are acquired by predicting the positions of the satellite and ground stations in time and
using these predictions to calculate a schedule for the system. The chapter also debates the merits
of the prediction technique and how this improves the volumetric throughput of the communications
system.

Chapter \ref{chp:COMMS_CONCEPT} presents the design of the SCSS. Initially, a functional
analysis is performed and then the high-level design is described. Two entities are introduced, the station
server and station handler and the design of each system with flow diagrams is discussed in detail.
Mechanisms used to ensure correct functionality within the limited resource environment are also presented.
Message formats are also discussed, along with the choice of markup language. The different messages
used by the system are also presented along with the structure of the file store. Another important part
of the SCSS is logging and the implementation of this is also discussed.

Chapter \ref{chp:IMP_TEST_PERFORM} discusses some implementation specific details of the SCSS.
The communications system is discussed on a program level and the other supporting software and
scripts are also discussed. The unique challenges faced when developing for an embedded system are
presented, which is followed by testing and performance descriptions. All tests performed on the system are
described and the importance of each test is explained. The chapter concludes by illustrating
memory and CPU performance figures achieved. These figures illustrate the low resource
usage of the SCSS.

Chapter \ref{chp:CONC} concludes the work with a discussion on contributions made and recommendations
regarding the path ahead as well as what sections require further investigation.

%\section{Summary}

%The history of the Leuven project was described along with where the department and this
%thesis fit into that project. The SCSS, the focus of this work, was presented along with the
%motivation for the system. This also includes the required functionality of a store-and-forward system.

%The research objectives and contributions were then presented. The objectives being the design, implementation
%and testing of  the SCSS, with store-and-forward functionality. Other contributions also included
%the link prediction algorithm, communications strategy, communications statistics, angle prediction and maximisation
%of the volumetric throughput of the communications system.

%Remote monitoring applications were presented as the main application for the LEO satellite communications
%system under investigation. Some examples were presented to illustrate the range of remote monitoring applications.
%The main characteristics of remote monitoring applications were mentioned and discussed to present a perspective
%from which to view the system design presented in later chapters.

%An overview of the work was also presented, which summarises the remaining chapters in this work.