\chapter{Introduction}
\label{chp:INTRO}


\section{Peer-to-Peer}

\subsection{Structured and unstructured P2P overlay networks}
\label{overlays}

P2P networks are created and maintained in the application layer of the Open Systems Interconnection (OSI) model protocol stack. This application
layer network is called the P2P overlay. Peers in an overlay network might have neighbours that have no relationship to their physical position in
the underlying network. Overlays can broadly be classified into structured and unstructured types. The classification is mostly based on the
differing methods of routing and content retrieval in the network. This section only provides a brief comparison between structured and unstructured
overlays. For a detailed comparison between the two types that also deals with many of the myths of structured overlays, please refer to
\cite{Castro_structured_overlay_myths}.

With unstructured approaches, one is never assured that a data item will be retrieved, even if that data item is present in the network. If many
duplicates of a data item are contained in the network, this becomes less of a problem, since it is assumed that the request will be routed to some
set of nodes that do  possess the item.

An unstructured architecture works well for content sharing and Voice over Internet Protocol (VoIP) networks, for example: P2P TV, BitTorrent,
Gnutella and Skype. The reason for this is the high level of duplication in these networks, especially for popular content. It is also easier to
perform keyword searches in unstructured networks and the overlay requires less maintenance.

Because there is no assurance that a data item might be retrieved from an unstructured network, especially when that item is scarce, unstructured
overlays are not considered adequate as a basis for P2P MMVEs, where all data items must be available at all times.

Structured overlays have been proposed that provide for efficient routing and reliable retrieval of data items. Some of these well known overlays
are: CAN \cite{CAN}, Chord \cite{chord}, Tapestry \cite{tapestry} and Pastry \cite{pastry}.

The basic idea of a structured overlay is that all nodes are identified by unique identifiers (IDs). A popular method to create the IDs is to use
hashes to a circular key space. Any node in the overlay network is then able to efficiently route a query with a given ID, to a node with an ID
closest to the given ID. An accurate comparison is that unstructured overlays are good at finding ``hay'', while structured overlays are good at
finding ``needles'' \cite{Rodrigues_acm_comms_p2p}.

%TODO: Add something about the effect of the hashing and why it's important

%%%%%%%%%%%%%%%%%%%%%%%%%%%%%%%%%%%%%%%%%%%%%%%%%%%%%%%%%%%%%%%%%%%%%%%
\section{P2P MMVEs}
\label{background}

Peer-to-Peer (P2P) Massively Multiuser Virtual Environments (MMVEs) have received significant attention from the research community, since
the first publication on the subject by Knutssonn et al. in 2004 \cite{knutsson_p2p_first}. P2P MMVEs promise to solve many issues prevalent in
today's Client/Server (C/S) based MMVEs. Some key issues have to be solved before P2P MMVEs can be implemented commercially. Over the past few years,
researchers have been addressing these challenges.

The remainder of this paper is structured as follows:
%
the field of P2P MMVEs is introduced in Section  \ref{p2p_network_architectures}, which contains an introduction to P2P overlays, the major advantages of P2P MMVEs and the key challenges that P2P MMVEs face.
%
An introduction to some classic consistency models, currently used in computer games, are introduced in Section \ref{classic_models}.
%
Section \ref{related_work} discusses some related surveys that have been completed in the field of P2P MMVEs and P2P applications in general.
%
Section \ref{p2p_MMVE_state_persistency} contains an analysis of different types of distributed storage for MMVEs. The section defines characteristics against which all storage schemes are evaluated. The section concludes with recommendations to implementers as to the applicability of the different types of storage to different types of games.
%
Section \ref{conclusion} concludes the paper by providing a brief summary and suggesting a number of areas for future work.

\section{Peer-to-Peer MMVE network architectures}
\label{p2p_network_architectures}

In 2004, an architecture using the peer-to-peer networking model to host MMVEs was proposed by Knutsson et al. \cite{knutsson_p2p_first}. This
revealed a new research field, which attempts to establish the P2P model as a viable alternative to the classic C/S and Client/MultiServer (C/MS)
architectures. This architecture does, however, still have a few major issues that need to be solved before MMVEs can be developed that use it. If
these issues, discussed in Section \ref{key_challenges}, can be solved, a P2P architecture holds some powerful advantages over a C/S system.

The core idea of the P2P model is that each peer contributes sufficient resources to the network to host itself. This also means that all functions
of the server in the classic C/S model are distributed amongst all peers. 

\subsection{Advantages}
\label{p2p_advantages}

There are various advantages to moving from C/S to P2P in MMVEs. These include: increased robustness, improved scalability, lower operator costs,
improved handling of transient player load and lower latencies.

The P2P system is robust, because there is no server that can fail, only individual peers. Individual peers failing will not affect any other peers
other than the peer that failed. This behaviour makes game down-time extremely unlikely.

Furthermore, because every peer hosts itself, the system is scalable. Another advantage is that no extra costs are incurred from an operator
perspective, when more peers join the network. This will also allow for efficient handling of transient loads. If many players suddenly enter the
game no resource provisioning issues will arise, since peers already possess their required resources.

P2P architectures also create a lot of opportunity for independent developers, because a large initial investment is no longer required to purchase
the expensive server hardware. Not only are hardware costs reduced, but running costs are also reduced. The bandwidth required by the game server is
now shared amongst users, which means that very little bandwidth costs will be incurred by the provider.

Latency is also improved, because it is now possible to directly communicate between peers and it is not necessary to communicate via a server. There
is also no single server that has to process game events. Game events need only be processed by other peers who find the specific events of interest.
The distribution of the load as well as direct communication will further reduce latency. Game events are defined in Section \ref{terminology}.

\subsection{Key challenges}
\label{key_challenges}

%State persistency, compared to state consistency
A key challenge with any networked game is how to maintain state consistency between root and replica objects. A root object is the authoritive
version of an object and the replica object is usually a local non-authoritive version. Root objects are usually found on the server and replica
objects are found in the local object cache of clients. The method by which the states between root and replica objects are updated is called the
consistency model. Solving the state consistency problem for P2P MMVEs is one of the major development challenges.

%Describe the challenge of state consistency

Another challenge for P2P systems is the required peer bandwidth. In a paper by Miller and Crowcroft, a packet simulator was created to determine the
required bandwidth and effective latency, if a game such as World of Warcraft were to be implemented using P2P technologies
\cite{Miller_p2p_infeasability}. Their simulation results indicate that today's networks are not able to host P2P MMVEs, with the required bandwidth
and latency constraints. Such a significant result requires verification, but at the least, it shows that reducing bandwidth and latencies for P2P
MMVEs should be a primary design requirement.

It should be noted that the overview presented here is only an overview of the different techniques and general trends present in the different areas
of peer-to-peer MMVEs. This overview does not presume to present an exhaustive list of papers in these areas, rather to place the topic of state
persistency in context; to show readers how state persistency fits into the context of peer-to-peer games and to allow readers to distinguish
between, for example, the topics of state persistency, interest management and event dissemination.

\subsubsection{State consistency}

\subsubsection{Cheating mitigation}
\label{key_challenges_cheating}

Cheating mitigation has been identified as a major issue for P2P systems \cite{knutsson_p2p_first}, \cite{challenges_p2p_gaming},
\cite{cheat_proof_event_ordering}. The challenges reside in the fact that peers are not under the control of the game producer. Since all server data
are distributed amongst peers, all peers have access to sections of the server data. Peers also have access to the distributed server code. One
advantage that can be exploited to prevent cheating is that no peer contains all server data and no single peer has more authority than another.

There are various security issues that are usually classified according to the level in the protocol stack where they occur. The areas identified by
\cite{cheat_proof_event_ordering} and expanded upon by \cite{cheating_taxonomy} are: game level, application level, protocol level and infrastructure level. This is consistent with the generally used layered security model \cite{distributed_systems_security}.

Game level cheats are ways in which a malicious player may gain an unfair advantage over other players, within the confines of the game. These cheats are usually because of software bugs and some examples are duplication and teleport cheats.

Application level cheats are where malicious players alter the game software to gain an unfair advantage. This is usually done by gaining access to
the game state to which they should not have access at the current time. An example of this is ``map reveal'' cheats in strategy games. Where the
``fog of war'' is removed and the player can observe all the opponent's movements. Other cheats are sometimes used that augment the player's UI with extra information that allows the player to make more informed decisions. It is debatable whether these additions are cheats. They are, however, considered almost essential for competitive WoW play.

Protocol level cheats are cheats based on the different methods of communicating data across the system. These usually concern dropping, delaying of modifying IP packets to achieve certain outcomes in the game. Infrastructure level cheats concern exploiting the underlying infrastructure on which the games are built. These include hacking the hardware or P2P overlay.


As with all taxonomies, all cheats may not cleanly fit into one if these boxes, some cheats may occur over multiple levels or a cheat with a specific outcome can be implemented differently on different levels. The field of P2P security has recently received more attention than in the past and has
started to bear fruit \cite{survey_p2p_game_cheats}. This is, however, an ongoing research field with many issues still open. For an in-depth review
of the security issues facing peer-to-peer system in general, refer to \cite{p2p_security_issues}. These issues are the same issues facing P2P MMVEs, with the exception of the game and application layer issues.

\subsubsection{Incentive mechanisms}

P2P schemes require all players to share resources in order to ensure correct functionality. The issue with this is that players might not want to share their resources, but still benefit from the resources of others. This is where incentive mechanisms become important. The function of these
mechanisms is to ensure that all players contribute resources, by incentivised contribution.

All distributed resource sharing models require incentive mechanisms. For example, Bittorrent systems use the tit-for-tat protocol to ensure that all
people downloading data are also contributing data \cite{tit_for_tat}. Such mechanisms are also required with P2P MMVEs. One advantage in designing
an incentive algorithm for a P2P MMVE is that players can be made to contribute resources for the duration of play. The issues with file sharing
systems are not present where a peer, after downloading a file, has no more incentive to contribute. When a peer plays a game, incentive can be
created to provide resources for the duration of the game.

Some incentive schemes proposed increase a player's reputation when resources are provided \cite{classic_p2p_reputation} \cite{proactive_reputation}. This might create a type of meta game, where players try to gain as much reputation as possible. It can however be argued that this scheme does not really enforce the provisioning of resources. A player who does not want to provide resources might not see a higher reputation as sufficient incentive to provide resources.

Other issues with incentive schemes is that sometimes players might have insufficient resources. Such players should be aided by other players with
sufficient resources and not be disallowed to play the game. When limited resources are taken into account, the issue of reporting a false amount of
available resources becomes a problem. A peer that has sufficient resources, might report insufficient resources, to not be penalised. It is evident
that there exists space for more research in this field.


\subsubsection{Distributed computation}

Non-Player Characters (NPCs) are characters that are not controlled by any human player, but are rather controlled by some artificial intelligence
routine or script executing on some host machine. These characters represent the traders and monsters in MMVEs and usually contain sets of rules that
determine how they should interact with Player Characters (PCs) as well as their own state information. An NPC's state can be how much money and
items it has to trade or how much health it still has after being attacked by a player.

In the original NPC host allocation classification by Fan, both NPC state and computational routines are combined into a single category
\cite{Fan_phd}. In the classification presented below, NPC state forms part of normal game state persistency, since NPC objects are game objects like
any other. The NPC routines requiring computational power are grouped under the heading of distributed computation. This heading is meant to include
the distribution of all in game computational elements.

Some game objects require computational power to function. An example of this is the Artificial Intelligence routines of NPCs or the computation of
physics effects on in-game objects. Some architectures assume that the computational requirements will be fulfilled where the object state is hosted
\cite{solipsis}, but other schemes exist that allow for the CPU power to be distributed amongst peers. One such scheme makes use of a ``job board''
like mechanism, where tasks are advertised on specialised super peers. Other peers monitor these super peers and may elect to perform the advertised
tasks \cite{fan_mediator_paper}.

\section{Objectives and contributions}
\label{objectives}


\section{Overview of this work}
\label{overview_intro}
