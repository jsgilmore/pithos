\chapter{Introduction}		%Kyk maar weer later na hierdie afdeling
\label{ch1}
\section{Background}

This project was proposed by Mr. John Gilmore,  who %to be done under the co-supervision of Mr J.\ Gilmore. 
%Mr. Gilmore 
is currently working on a project that aims to create an improved, state persistent architecture to be implemented in peer-to-peer(P2P) massively multiplayer online role-playing games (MMORPGs), that would be more effective, use less bandwidth and cause less latency than the current architectures in place~\cite{gilmore}. Persistency here refers to the game data on all the clients being consistent, especially when it comes to  game events. There are still many problems with the implementation of a P2P massively multiplayer online game (MMOG), but one of the biggest problems is state percistency. %To be able to create a solution to this problem, research is required into the characteristics of data stored by MMOGs. The best way to characterise the data stored by MMOGs is to gain access to and analyse the database of an MMOG. 

Another challenge for P2P systems is the peer bandwidth required. \citet{p2p} found that today's networks are not able to host P2P MMOGs with the required bandwidth and latency constraints currently in place. This was concluded by creating a packet simulator to implement P2P communication in a popular MMORPG called World of Warcraft (WoW)~\cite{p2p}. This result needs verification, but indicates that reducing bandwidth and latency should be an important design requirement for P2P MMOGs.

To be able to find possible solutions to the problems that currently face the P2P MMOG architectures, research is required into both the general characteristics of Client/Server (C/S) type MMOGs, and into the characteristics of data stored by C/S MMOGs~\cite{gilmore}. This research will see the creation of models that show how frequently game objects are stored as well as the size of these objects. The models can then be used in order to determine the performance that is required for P2P MMOG storage mechanisms. Models of player movement in the game also needs to be created to better understand to what extent different peers move together and to model how the data must be spread across the system. 

In order to create models of player movement in MMOGs, a tool is required that is able to collect player location data on the client side of an MMOG that is currently populated by many players. This project is about the creation of software that can effectively and accurately capture the location data of players in an MMOG. This data can then be used to create movement models of players that can in turn be used in creating a persistent P2P MMOG architecture. 

%With the current system there is only one server with a big database, and all the players have to log into that server to be able to play the game. Since this server could be on the other side of the world, latency can become a problem. The system he is working on would be distributed across the computers of players currently in the same area, and will potentially cause much less latency and thus faster gameplay. It would also have less traffic to a specific server, which would further increase speed.

%In order to be able to design such a system, information needs to be gathered on the movement of players, so that patterns can be looked for and mathematical models created that could simulate player movements. These models of player movement are necessary to create the distributed database system, because how the data is stored on the computers of different players would depend on how long they typically move together and so forth. Before any of this can be done however, a method to capture and log player movements in a simulated game world environment is first needed, which is where this project comes in. 

For this project, World of Warcraft (WoW) was chosen as the best game to use to capture data from, because it is currently the largest MMORPG in the world~\cite{subscription}. A program needs to be written that is able to get the location information in coordinates of the players in this game world, and save it in a file to be used for further analysis. A program that can read and visually display these logs for a quick and convenient analysis is also needed. The accuracy of this movement capturing and displaying software will be critically analysed in this report. %moet dalk se dit was nie nodig nie? "It was decided that this program would be beneficial" of so iets.

This software will be a valuable tool for gathering enough movement data to better understand how players move in MMOGs and to create movement models to simulate this behaviour when testing new P2P MMOG architectures. Research into the characteristics of data stored by C/S MMOGs is still needed however. Since access to the real WoW servers are not possible, the research is done by analysing the database of an open source private WoW server called ArcEmu. This server uses a MySQL database and emulates the functions of the real WoW servers well enough that the original client software of WoW can be used with it. The database of ArcEmu will be analysed to determine the frequency that game objects are stored at and the average size of these database queries. This will give a better understanding of the performance required from MMOG storage mechanisms. The data will be useful in improving on the current P2P MMOG architectures and will provide a benchmark of the required storage performance for a mature and established MMOG.

%In addition to player tracking software, knowledge on how the database on the server-side of WoW functions is also needed. ArcEmu is an open source private WoW server, which uses a MySQL database and simulates the functions of the real WoW server. Since it is not possible to access the database of the real WoW server, an ArcEmu server will be set up and the data throughput and amount of queries made to the database will be analysed to better understand how the current database system of WoW works. This will allow the automation of real world traffic to be modelled and used to design and test a distributed database system for use in MMORPG's. The effectiveness of such a system will be determined by using models based on real world traffic to test it.

\section{Aims}

The aim of this project is to start the first step in a larger field of research, where a state persistent architecture for P2P MMOGs will be implemented. Before this larger project can be started, research into the characteristics of the data storage methods used in MMOGs is required. Models of player movement in MMOGs are also required to test the performance of the P2P architecture, but in order to create those models a tool is first needed that can capture the location data of players in an MMOG accurately and efficiently. 

With the above goal in mind, this project aims to create software that is able to extract all the location data of human players that is sent to the WoW client software by the WoW server. The location data of any player within  a certain radius of the local player is usually sent to the client. The software must be able to identify each human player uniquely and accurately extract, log and save the location data for each different player in a separate log file. The software must also be able to display movement traces of players accurately to create a  picture of the combined movement data.

Another program needs to be created that can read the log files created by the player tracking software, to recreate a picture of the movement data captured. This program should then be able to export the movement traces as a bitmap image file of which the size is specified by the user. This would allow the program to display larger sets of movement data that would not normally fit into the display window. A zoom function would further enlarge the area that can be displayed. 

The final part of the project aim is to monitor and analyse the database of a private WoW server in order to better understand the data storage requirements of an MMOG. This analysis includes the frequency of queries made to the database as well as the size of the largest queries. %The analysis will also investigate the average amount of queries generated per player and the data throughput of the database.
The data can then be used as a benchmark of data storage requirements for a mature MMOG which will guide the creation of a new architecture for P2P MMOGs.


 %Each player must be identified uniquely and all the location information available must be logged and saved for later use. This information will be saved in a different text file for each uniquely identifiable player. The software will be tested to ensure the accuracy of the data extracted. A program that can read the logs and visually display the movements of characters for convenient and quick analysis will also be written. Furthermore the database of a private WoW server will be analysed by creating a private server and logging into it with several accounts, and playing the game, to create realistic game traffic to be analysed. The analysis will investigate the average amount of queries generated per player and the data throughput of the database.

\section{Literature Study}
%\subsection{Previous work}
The most comprehensive previous work done in the same field as this project is the work done by~\citet{previous}, where they collected and analysed avatar location data in WoW Battlegrounds. Their research was done in late 2008, before WoW started using Rivest Cipher 4 (RC4) encryption on their packets, which allowed them to capture network traffic with Microsoft Network Monitor 3.3 and analyse the captured traffic for location data sent to the WoW client. The movement data was only extracted after the network traffic capturing was completed. This method is in contrast with the current project where location data will be extracted in real time from the client software.

The goal of their analysis was also to produce movement models of avatars, but they only focused on player activity in a battleground environment. This is different from the current project where the goal is to write a program that can gather location data effectively and in real time from anywhere in the game world, to allow the user to choose from where to capture location data. Their analysis concluded that players move in a hotspot-based model in WoW Battlegrounds.


Several other projects have also been done where the location data of players have been extracted~\cite{wradar, wowbot, wowradapp, wowradar}. In these projects the only reason for gathering the location data of players was for displaying purposes.  There are several bots and radar applications that have been written to display the position of other characters in relation to the position of your own character. None of these programs have the ability to log the location data, or to show movement traces of players. The sources listed here are also projects done on earlier versions of WoW, but their methodology is very similar to the one used in this project.

%Write about previous tracking work, also mention bots and other programs written that can get location information. None of these programs actually log the information though.

\section{Summary}

Chapter~\ref{ch1} gives an introduction to the project, discusses previous work done on similar projects and gives a brief summary of the content of the text. Chapter~\ref{ch2} gives background information on MMORPGs, WoW and the private server called ArcEmu. This information is needed to understand the terms, methods and results discussed in later chapters. Chapter~\ref{database} describes the tools used and the methodology followed to analyse the database of a private WoW server. This information is relevant when results are discussed later. Chapter~\ref{ch5} discusses the data structure used by WoW to store player data in, as well as how the tracking software will be written to extract the necessary data from the client software's memory. Chapter~\ref{results} discusses the tests and analysis that will be done and goes on the reveal and discuss the results obtained from the analysis and tests, comparing them to expected results. Chapter~\ref{conclusion} lists recommendations for improvements to be made to the software and makes conclusions based on the results from chapter~\ref{results}.
%Summaries of all the chapters. Will be added last.