\documentclass[journal,oneside,a4paper,onecolumn]{IEEEtran}

% Some very useful LaTeX packages include:
% (uncomment the ones you want to load)

% *** CITATION PACKAGES ***
%
\usepackage{cite}
% cite.sty was written by Donald Arseneau
% V1.6 and later of IEEEtran pre-defines the format of the cite.sty package
% \cite{} output to follow that of IEEE. Loading the cite package will
% result in citation numbers being automatically sorted and properly
% "compressed/ranged". e.g., [1], [9], [2], [7], [5], [6] without using
% cite.sty will become [1], [2], [5]--[7], [9] using cite.sty. cite.sty's
% \cite will automatically add leading space, if needed. Use cite.sty's
% noadjust option (cite.sty V3.8 and later) if you want to turn this off.
% cite.sty is already installed on most LaTeX systems. Be sure and use
% version 4.0 (2003-05-27) and later if using hyperref.sty. cite.sty does
% not currently provide for hyperlinked citations.
% The latest version can be obtained at:
% http://www.ctan.org/tex-archive/macros/latex/contrib/cite/
% The documentation is contained in the cite.sty file itself.


% *** GRAPHICS RELATED PACKAGES **
%
  \usepackage{graphicx}
  \graphicspath{{../Figures/}}
  \DeclareGraphicsExtensions{.pdf,.png}
  \usepackage{color}

% *** MATH PACKAGES ***
%
\usepackage[cmex10]{amsmath}
% A popular package from the American Mathematical Society that provides
% many useful and powerful commands for dealing with mathematics. If using
% it, be sure to load this package with the cmex10 option to ensure that
% only type 1 fonts will utilized at all point sizes. Without this option,
% it is possible that some math symbols, particularly those within
% footnotes, will be rendered in bitmap form which will result in a
% document that can not be IEEE Xplore compliant!
%
% Also, note that the amsmath package sets \interdisplaylinepenalty to 10000
% thus preventing page breaks from occurring within multiline equations. Use:
%\interdisplaylinepenalty=2500
% after loading amsmath to restore such page breaks as IEEEtran.cls normally
% does. amsmath.sty is already installed on most LaTeX systems. The latest
% version and documentation can be obtained at:
% http://www.ctan.org/tex-archive/macros/latex/required/amslatex/math/

%\usepackage{amssymb}%............................ AMS Symbol fonts



% *** SPECIALIZED LIST PACKAGES ***
%
%\usepackage{algorithmic}
% algorithmic.sty was written by Peter Williams and Rogerio Brito.
% This package provides an algorithmic environment for describing algorithms.
% You can use the algorithmic environment in-text or within a figure
% environment to provide for a floating algorithm. Do NOT use the algorithm
% floating environment provided by algorithm.sty (by the same authors) or
% algorithm2e.sty (by Christophe Fiorio) as IEEE does not use dedicated
% algorithm float types and packages that provide these will not provide
% correct IEEE style captions. The latest version and documentation of
% algorithmic.sty can be obtained at:
% http://www.ctan.org/tex-archive/macros/latex/contrib/algorithms/
% There is also a support site at:
% http://algorithms.berlios.de/index.html
% Also of interest may be the (relatively newer and more customizable)
% algorithmicx.sty package by Szasz Janos:
% http://www.ctan.org/tex-archive/macros/latex/contrib/algorithmicx/

% *** ALIGNMENT PACKAGES ***
%
\usepackage{array}
% Frank Mittelbach's and David Carlisle's array.sty patches and improves
% the standard LaTeX2e array and tabular environments to provide better
% appearance and additional user controls. As the default LaTeX2e table
% generation code is lacking to the point of almost being broken with
% respect to the quality of the end results, all users are strongly
% advised to use an enhanced (at the very least that provided by array.sty)
% set of table tools. array.sty is already installed on most systems. The
% latest version and documentation can be obtained at:
% http://www.ctan.org/tex-archive/macros/latex/required/tools/


\usepackage{mdwmath}
\usepackage{mdwtab}
% Also highly recommended is Mark Wooding's extremely powerful MDW tools,
% especially mdwmath.sty and mdwtab.sty which are used to format equations
% and tables, respectively. The MDWtools set is already installed on most
% LaTeX systems. The lastest version and documentation is available at:
% http://www.ctan.org/tex-archive/macros/latex/contrib/mdwtools/

% IEEEtran contains the IEEEeqnarray family of commands that can be used to
% generate multiline equations as well as matrices, tables, etc., of high
% quality.

% *** SUBFIGURE PACKAGES ***
% subfig.sty, also written by Steven Douglas Cochran, is the modern
% replacement for subfigure.sty. However, subfig.sty requires and
% automatically loads Axel Sommerfeldt's caption.sty which will override
% IEEEtran.cls handling of captions and this will result in nonIEEE style
% figure/table captions. To prevent this problem, be sure and preload
% caption.sty with its "caption=false" package option. This is will preserve
% IEEEtran.cls handing of captions. Version 1.3 (2005/06/28) and later
% (recommended due to many improvements over 1.2) of subfig.sty supports
% the caption=false option directly:
\usepackage[caption=false,font=footnotesize]{subfig}
%
% The latest version and documentation can be obtained at:
% http://www.ctan.org/tex-archive/macros/latex/contrib/subfig/
% The latest version and documentation of caption.sty can be obtained at:
% http://www.ctan.org/tex-archive/macros/latex/contrib/caption/

%Setting captions to centered (Not IEEE journal standard)
\makeatletter
\long\def\@makecaption#1#2{\ifx\@captype\@IEEEtablestring%
\footnotesize\begin{center}{\normalfont\footnotesize #1}\\
{\normalfont\footnotesize\scshape #2}\end{center}%
\@IEEEtablecaptionsepspace
\else
\@IEEEfigurecaptionsepspace
\setbox\@tempboxa\hbox{\normalfont\footnotesize {#1.}~~ #2}%
\ifdim \wd\@tempboxa >\hsize%
\setbox\@tempboxa\hbox{\normalfont\footnotesize {#1.}~~ }%
\parbox[t]{\hsize}{\normalfont\footnotesize \noindent\unhbox\@tempboxa#2}%
\else
\hbox to\hsize{\normalfont\footnotesize\hfil\box\@tempboxa\hfil}\fi\fi}
\makeatother


% *** FLOAT PACKAGES ***
%
\usepackage{fixltx2e}
% fixltx2e, the successor to the earlier fix2col.sty, was written by
% Frank Mittelbach and David Carlisle. This package corrects a few problems
% in the LaTeX2e kernel, the most notable of which is that in current
% LaTeX2e releases, the ordering of single and double column floats is not
% guaranteed to be preserved. Thus, an unpatched LaTeX2e can allow a
% single column figure to be placed prior to an earlier double column
% figure. The latest version and documentation can be found at:
% http://www.ctan.org/tex-archive/macros/latex/base/

% *** PDF, URL AND HYPERLINK PACKAGES ***
%
\usepackage{url}

\usepackage{sistyle}
    \SIstyle{S-Africa}
    \SIunitspace{{\cdot}}
    \SIunitdot{{\cdot}}

% generate nice bookmarks and hyperrefs when exporting to pdf and dvi (screen version):
\usepackage[a4paper,plainpages=false,colorlinks,linktocpage,bookmarks=true,bookmarksopen=false]{hyperref}
% use this for printing only (no color, print version):
%\usepackage[a4paper,plainpages=false,colorlinks=false,linktocpage,bookmarks=true,bookmarksopen=false]{hyperref}

% correct bad hyphenation here
\hyphenation{op-tical net-works semi-conduc-tor}

%List of acronyms used in text
 \usepackage{acronym}%.......................... Acronym package to handle acronyms in text

\acrodef{MMOG}{Massively Multiplayer Online Game}
\acrodef{MMORPG}{Massively Multiplayer Online Role Playing Game}
\acrodef{WoW}{World of Warcraft}
\acrodef{MUD}{Multi-User Dungeon}
\acrodef{PvP}{Player-versus-Player}
\acrodef{P2P}{Peer-to-Peer}
\acrodef{CS}[C/S]{Client/Server}
\acrodef{CMS}[C/MS]{Client/Multi-Server}
\acrodef{NPC}{Non-Player Character}
\acrodef{aoi}[AoI]{Area of Interest}
\acrodef{alm}[ALM]{Application Level Multicast}
\acrodef{ui}[UI]{User Interface}
\acrodef{DHT}{Distributed Hash Table}

\begin{document}

%
% paper title
\title{Massively Multiplayer Online Game Proposal}

\author{\IEEEauthorblockN{John S. Gilmore\\}
\IEEEauthorblockA{MIH Media Lab\\
Department of Electronic Engineering\\
Stellenbosch University\\
Stellenbosch, South Africa\\
Email: jgilmore@ml.sun.ac.za}}

% make the title area
\maketitle

\hfill August, 2010

\section{Introduction}


\IEEEPARstart{W}{hen} deciding on an \ac{MMOG} to recommend for hosting in South Africa, the advantages of a locally hosted \ac{MMOG} should first be
explored. We feel that the two main advantages to a locally hosted \ac{MMOG} are significantly reduced latencies and access to a local community. The
recommended game should, therefore, exploit these two advantages to be successful in the South African context. A game that fully exploits these two
advantages, will probably posses the greatest traction when players join the local servers.

\section{Localisation}

The greatest advantage of a locally hosted MMOG, other than the reduced latencies, which every game will have, is the local community. If an MMOG can
be launched in South Africa with sufficient media hype, it could attract a strong local community. This community should be leveraged, to further
stimulate game growth.

An effective way by which the community may be leveraged is by running nation wide tournaments. This can be done at school level, university level
and provincial level. Tournaments, with ranking and prizes are a great way to stimulate growth in a game and build a national gaming culture.
Stimulation at this level is not possible for other, internationally hosted, games. For these tournaments to work, the chosen game should possess the
qualities to provide for an engaging experience when playing with others.

It is believed that games that focus more on story driven play and solo play will not work as well, because there is no way to leverage the local
community. That is not to say that if a predominantly solo game is locally hosted it will not attract players, all players already playing online
will probably move to the local server, purely because of the reduced latency. The issue is that it would be more difficult to obtain new players as
the solo player has no motivation to invite new players into the game.

In \ac{PvP} oriented games, the game becomes more fun to play as new players join the game. \ac{PvP} oriented games usually focus on having players
fight against each other for rewards. Players usually fight in groups, called Guilds. Guilds provide social ``clubs'' for players to belong to.
Players' loyalty to their guild are high and it is known that some players continue to play a game, not because they still enjoy it that much, but
because of the pressure and also recognition from their guild. Building successful guilds, therefore, will provide for a strong social network that
will ensure the success of a game.

This of course also requires a large group of players to start with. If there are no players and no guilds, a \ac{PvP} oriented game is sure to fail.
This again reinforces the need for the significant initial media coverage to obtain a sufficient number of players to achieve the ``critical mass''
of the game.

\section{Recommendation}

\begin{table}[htbp]
\centering
\begin{tabular}{|l|c|}
  \hline
  Game & Number of Guilds \\
  \hline  \hline
  World of Warcraft & 411 \\
  \hline
  Warhammer Online & 238 \\
  \hline
  Age of Conan & 196 \\
  \hline
  Aion & 163 \\
  \hline
  Lord of the Rings Online & 152 \\
  \hline
  Star Wars: The Old Republic & 117 \\
  \hline
  Guild Wars & 106 \\
  \hline
  Darkfall & 92 \\
  \hline
  EVE Online & 87 \\
  \hline
\end{tabular}
\caption{Ten MMOGs with the highest number of registered guilds} \label{tab_guilds}
\end{table}
%
With the requirement of a social or guild oriented game, a study was undertaken to see what the MMOGs are with the largest number of guilds. This
list was constructed from the guild list of www.mmorpg.com, a popular information portal for MMOGs. The top ten games are shown in Table
\ref{tab_guilds}.

From Table \ref{tab_guilds}, it is clear that \ac{WoW} is the most popular MMOG by far. This matches the 11 mil player subscription count of the
game. The issue with hosting \ac{WoW} is the prohibitive cost. After speaking to some people in the MIH group, it is clear that the game would just
cost too much to host for the reportedly 70,000 South African players on the European server. This is mostly due to the licensing fees that Blizzard
require.

The second most popular game is Warhammer Online. The game is very similar to \ac{WoW}, but with a greater focus on \ac{PvP}, making it more part of
the everyday ``life'' of the player. The game contains mechanics like public quests and Realm-versus-Real combat to make PvP an integral part of the
game. It also possesses many advanced guild features to incorporate guilds into the game and to make being part of a guild a much more fulfilling
experience. Warhammer is not only a game with a high number of guilds, but also a well made, highly polished game, that is a lot of fun to play. The
Warhammer IP is also well known and loved by players and the game creators were able to draw from a rich set of lore to make the world come to life.

The author is of the opinion that Warhammer Online would be a very suitable candidate for a locally hosted MMOG. Some alternative choices are
discussed in Section \ref{alternatives} that might also be good choices depending on their success after launch.

It is also proposed that if Warhammer Online is used, some official local guilds should be created that can take part in tournaments. One ladder
might for example be all South African tertiary institutions. The tertiary institutions can have recruiting drives to recruit members to their guilds
and host real social events for guild members. This will bring the guild into the physical world and provide a great social platform for guildies to
socialise on. One can then have guild versus guild tournaments to determine the best guild with prizes for all active guild members. This real world
link with the game works very well with a game that is very focused on guild play. It is the belief of this author that such a physical link with the
game is required, to make it successful locally and to improve the gaming culture in the country.


\section{System requirements}

The minimum system requirements for Warhammer Online are the following:

\textbf{Windows XP}:
\begin{itemize}
    \item 2.5 GHz P4 processor or equivalent
    \item 1 Gigabyte RAM
    \item A 128 MB Video Card, with support for Pixel Shader 2.0
    \item At least 15 GB of hard drive space
\end{itemize}

\textbf{Windows Vista}:
\begin{itemize}
    \item 2.5 GHz P4 processor or equivalent
    \item 2 Gigabyte RAM
    \item A 128 MB Video Card, with support for Pixel Shader 2.0
    \item At least 15 GB of hard drive space
\end{itemize}

\textbf{Mac OS X}:
\begin{itemize}
    \item Mac OS X 10.5.7
    \item Intel Core Duo Processor
    \item 2 Gigabyte RAM
    \item ATI X1600 or NVidia 7300 GT with 128 MB VRAM
    \item At least 15 GB of hard drive space
\end{itemize}

From this list is can be seen that the game has very moderate system requirements and should work on most PCs still in households today. As an added
bonus, the game also has a Mac client which increases the pool of eligible players. It should be noted that this is not an old game. It was released
at the end of 2008, but at that time, the graphics were already thought to be a bit dated. The graphics are however much better than that of \ac{WoW}
and the lower graphics is exactly what allows for the lower system requirements.

\section{Cost and fees}

The Warhemmer Online game client has to be purchased, before a valid account may be created. The game client is, however, not expensive anymore and
can now be bought from www.kalahari.net for R120.48, including shipping. This is much less than the standard price of R350 for all new releases.

Warhammer Online also requires a monthly subscription fee, payable by credit card or game time card. When paying month-to-month, this fee is \$15 per
month, which is the standard subscription fee for subscription-based MMOGs. Interestingly, the first game in the top MMOG list in Table
\ref{tab_guilds}, to use an item mall (micro transactions) as a form of payment, is ``Perfect World Online'' at number 14. All the games preceding it
either have a monthly subscription fee of \$15 or are free to play.

\section{Alternatives}
\label{alternatives}

As with all things technological, there are a few very promising MMOGs on the horizon. These are Guild Wars 2 and Star Wars: The Old Republic. It is
interesting to note that, while not released yet, Star Wars: The Old Republic is 7th in the list in Table \ref{tab_guilds}, with 117 active guilds.
This just shows the power of the Star Wars IP and is something that will undoubtedly lead to a large initial influx of players to investigate whether
the game stands up to all the very high expectations. The high expectations for Star Wars are also not unfounded.

The developer, Bioware, has produced some very successful titles in the Role Playing Game genre. These include the Baldur's Gate series, the Icewind
Dale series and Neverwinter Nights, which are all based on the existing Dungeons and Dragons rules. Recently, they also produced Dragon Age: Origins
and Mass Effect which were also a huge successes. What does remain to be seen, is how well they cope with the extra technical expertise required to
design a massive persistent world for an MMOG. This author does not doubt that the world will be vibrant and well designed, but where the game will
stand or fall will be on its technical implementation of the network architecture and how well Bioware's story telling expertise will cope with the
persistent world that they now have to manage.

The only reason why Star Wars cannot be recommended yet, is because it has not been released and as was seen with Star Trek Online, even an MMOG with
a great IP can fail if the gameplay is buggy and bland. It is also not known how much focus there will be on PvP and guilds in the game. With
Bioware's strong focus on storytelling, this is something that might be seen as secondary to the purpose of the game.

Another very promising game is Guild Wars 2. While not a known IP, Guild Wars 1 is a very successful MMOG that incorporated many novel ideas about
MMOG design into a well polished game that is also \emph{free to play}. The sales structure of Guild Wars is to bring out expansions regularly and
use the money generated from the expansions to fund the development of the game. Depending on how local revenue will be generated, this might however
be an issue.

NCSoft, the publisher of Guild Wars 2 have a lot of experience publishing MMOGs. Popular MMOGs they have published include: Aion, Blade and Soul,
City of Heroes/Villains, Guild Wars, Lineage and Tabula Rasa. The developer of Guild Wars 2, ArenaNet also developed the original Guild Wars series
that included three stand alone titles and one expansion.

The marketing material of Guild Wars 2 looks very promising and from interviews with the developers, it is clear that they are trying out some
interesting mechanics and some changes to the standard MMOG formula. This is something the industry needs at this point, with too many WoW clones
being made and too little in the way of innovation, because of the large risk involved in developing MMOGs.

\section{Conclusion}


This proposal reviewed what the advantages were of locally hosting an MMOG in South Africa. It found that the decreased latency and the ability to
leverage a local market and community are the greatest advantages. To harness these advantages, it is important to choose a game that will benefit
from the localisation of the game.

Warhammer Online was proposed as such an MMOG, because of the strong focus on player interaction, player conflict and player communities. In order to
harness the local hosting of an MMOG, it was proposed to tightly couple the game with physical world events, tournaments and ladders. Also to
stimulate the uptake of the game at school and university level by creating official chapters and promoting physical world social interactions within
these structures.

Both the system requirements and cost of Warhammer Online were shown to be very reasonable, making the game accessible to a large audience.

Finally, Guild Wars 2 and Star Wars: The Old Republic were put forward as promising alternatives, with the caveat that one should make sure of the
final product before it is decided to host the game locally as it is always possible that an MMOG will fail catastrophically.

% that's all folks
\end{document}
